\documentclass[../main.tex]{subfiles}

\begin{document}

\section{Topological Yang-Mills theories}\label{sec:tym}

Let spacetime be $\mathbb{R}^{4}_{\infty}$, endowed with a globally flat metric $\mathrm{g}$. The 4-manifold $\mathbb{R}^{4}_{\infty}$ is the one-point compactification of the standard $\mathbb{R}^4$, compatible with its conformal structure at infinity. This is a convenient choice of background for TYM, because it is compact, Riemannian and 4-dimensional --- necessary conditions for the rigorous definition of the YM instanton\footnote{$\mathbb{R}^{4}_{\infty}$ is also homeomorphic, but not diffeomorphic, to $S^4$. Additionally, $\mathrm{g}$ is conformally equivalent to a round metrics on the 4-sphere. Thus, CFTs on $\left( \mathbb{R}^{4}_{\infty}, \mathrm{g}\right)$ can pick up non-trivial topological effects while conveniently defined on a flat background.}. Due to the nature of topological field theories, it is convenient to work with differential forms. For future reference, $\wedge$ is the exterior product, $d$ is the exterior derivative, $v \rfloor$ is the interior product along the smooth vector field $v$, and $\star$ is the Hodge dual defined with respect to $\mathrm{g}$. All these operations occur on the exterior algebra $\bigwedge \mathbb{R}^{4}_{\infty}$ of spacetime.

Let $SU(N)$ with Lie algebra $\mathfrak{su}(N)$ be our structure group. This is a physically relevant choice, but also mathematically convenient as $SU(N)$ instantons are reasonably well-understood\footnote{The moduli of irreducible $SU(N)$ instantons is finite-dimensional, generically smooth and Kähler. Additionally, $SU(N)$ instantons can be fully classified via the ADHM construction.}. Also for future reference, we define on each event $x \in \mathbb{R}^{4}_{\infty}$ the $\mathbb{Z}_2$ graded Lie bracket $[X,Y] \equiv X \wedge Y-{(-1)}^{|X||Y|}Y \wedge X$, where $|X|$ is the grading (statistics) of $X$, the adjoint representation $\mathrm{ad}_X Y \equiv \left[X, Y\right]$, and the Killing form $\kappa \left(X,Y\right) \equiv \tr \left( \mathrm{ad}_{X} \mathrm{ad}_{Y} \right)$ acting on $X, Y \in \mathfrak{su}(N) \otimes \bigwedge_x \mathbb{R}^{4}_{\infty}$.

The global framework of an $SU(N)$ gauge theory is that of the universal bundle $[ SU(N)\times\mathcal{G} ] \hookrightarrow ( P \times \mathcal{A} ) \rightarrow ( \mathbb{R}^{4}_{\infty} \times \mathcal{M} )$~\cite{baulieu1985a,baulieu1988a}. The smooth  $(N^2+3)$-manifold $P$ is part of the more commonly known principle bundle structure $SU(N) \hookrightarrow P \rightarrow \mathbb{R}^{4}_{\infty}$, together with its adjoint bundles, $ \mathrm{Ad} P \equiv P \times_{SU(N)} SU(N) $ and $ \mathrm{ad}P \equiv P \times_{SU(N)} \mathfrak{su}(N)$. The space of all smooth $SU(N)$ connections on $ P $ is $ \mathcal{A} \equiv C^{ \infty } ( J^1 P ) $, where $ J^1 P $ is its 1st jet bundle, and the space of all smooth gauge transformations is $ \mathcal{G} \equiv C^{ \infty } ( \mathrm{Ad}P ) $, where $ \mathrm{Lie} (\mathcal{G}) \equiv C^{ \infty } ( \mathrm{ad} P ) $ is its Lie algebra. Finally, $ \mathcal{M} \equiv \mathcal{A}/\mathcal{G}$ is the gauge moduli\footnote{The definition of a universal bundle requires a free action of $SU(N) \times \mathcal{G}$ on $P \times \mathcal{A}$. Thus, it is implicit that we are only considering framed irreducible connections, preventing singularities in $\mathcal{M}$ and the ill-posedness of its cohomological problem.}.

A gauge field $A(x)$ is defined as an element in $C^\infty ( \mathrm{ad}P \otimes \bigwedge^1\mathbb{R}^{4}_{\infty} )$, which results from a pullback to spacetime of an $SU(N)$ connection living on $P$. Similarly, a universal gauge field $\tilde{A}(x)$ is defined as the result of a pullback to $\mathbb{R}^{4}_{\infty} \times \mathcal{M}$ of a universal $[SU(N) \times \mathcal{G}]$ connection living on $P \times \mathcal{A}$, and can be written as
\begin{equation}
  \label{eq:universal-connection}
  \tilde{A} = A + c \;,
\end{equation}
where $c(x) \in C^\infty ( \mathrm{ad}P \otimes \bigwedge^0\mathbb{R}^{4}_{\infty} )$ is a local projection of the Maurer-Cartan form on $ \mathrm{ Lie } (\mathcal{G})$ --- the well-known Faddeev-Popov ghost field. The universal exterior derivative $\tilde{d}$ can also be written accordingly,
\begin{equation}%
  \label{eq:universal-exterior-derivative}
  \tilde{d} = d + s \;,
\end{equation}
where $s$ is the local projection of the exterior derivative on $ \mathrm{Lie} (\mathcal{G})$ --- the well-known BRST operator. The $\mathbb{Z}_2$ grading we mentioned earlier is actually a $\mathbb{Z}_2$ bi-grading that exists in the exterior algebra on $P \times \mathcal{A}$, and which descends to spacetime as a differential form rank, and a ghost number. Thus, equations~\eqref{eq:universal-connection} and~\eqref{eq:universal-exterior-derivative} is just a bi-grading split into \mkbibquote{components} $(1,0)$ and $(0,1)$. In particular, they give the meaning of an 1-form with ghost number 0 to $A$, and a 0-form with ghost number 1 to $c$. Accordingly, the statistics of $A$, and $c$, is odd (fermionic) --- check Table~\ref{tab:gradings} for reference. Finally, $\tilde{d}$, $d$, and $s$ are all nilpotent operators by definition, and~\eqref{eq:universal-exterior-derivative} resumes to $sd+ds=0$, which implies $d$ and $s$ are fermionic.

The universal curvature $ \tilde{F} $ of $\tilde{A}$ is given by\footnote{From now on, whenever the context is sufficiently clear, we will ommit $\wedge$. In particular, $X \wedge X = XX = X^2$.}
\begin{subequations}%
  \label{eq:universal-curvature}
  \begin{align}
    \tilde{F} & \equiv \tilde{d}\tilde{A}+\tilde{A}^2 \;, \\
              & = F+\psi+\phi \;,
  \end{align}
\end{subequations}
and it has the curvature $F\equiv dA+A^2$ of $A$ as the $(2,0)$  component, the so-called \emph{topological ghost field} $\psi \equiv sA + Dc$ as the $(1,1)$ component, and the \emph{2nd generation topological ghost field} $\phi \equiv sc + c^2$ as $(0,2)$ component. We must clarify $D \equiv d+ \left[A, \hphantom{A}\right] $ is the exterior covariant derivative in the adjoint representation.

The Bianchi identities for $\tilde{F}$, and $F$, give
\begin{subequations}%
  \label{eq:universal-bianchi-identity}
  \begin{align}
    \tilde{D}\tilde{F}          & =0 \;,  \\
    sF+D\psi+ \left[c, F\right] & = 0 \;.
  \end{align}
\end{subequations}
And, from them, we can read the TYM BRST symmetries~\cite{baulieu1988a},
\begin{subequations}\label{eq:tym-brst}
  \begin{align}
    sA    & = -Dc + \psi \;,                     \label{eq:tym-brst-A}    \\
    sc    & = - c^2 + \phi \;,                   \label{eq:tym-brst-c}    \\
    s\psi & = -D\phi -  \left[c, \psi\right]  \;,\label{eq:tym-brst-psi}  \\
    s\phi & = - \left[ c, \phi \right]  \;,       \label{eq:tym-brst-phi} \\
    sF    & = -D\psi - \left[ c, F \right]  \label{eq:tym-brst-F} \;.
  \end{align}
\end{subequations}
This is the most general realization of the BRST algebra, capturing the full cohomological structure of traditional gauge theories\footnote{It is possible to augment this algebra, and the universal bundle construction, by considering the anti-BRST transformations~\cite{birmingham1991b,carvalho1992a,perry1993a}. However, no extra cohomological information is gained.}. In mathematical language,~\eqref{eq:tym-brst} is called the \emph{BRST/Kalkman model} of the $\mathcal{G}$-equivariant cohomology of $\mathcal{A}$. Imposing $c=0$ in~\eqref{eq:tym-brst-A}, \eqref{eq:tym-brst-psi}, and~\eqref{eq:tym-brst-F}, we get the so-called \emph{Weil model}, and imposing $c=0$ everywhere, we get the \emph{Cartan model} --- which is Witten's original \textquote{BRST-like supersymmetry}~\cite{witten1988d,ouvry1989a,kanno1989a,kalkman1993a}. Most importantly, these models are isomorphic to each other in the subspace of basic forms~\cite{baal1990a,witten1991a,cordes1995a}. Physically,~\eqref{eq:tym-brst} represents a much stronger set of symmetries than the usual YM BRST from particle physics. The latter can be obtained from the former by imposing the so-called \emph{horizontal conditions}, $\psi=\phi=0$.

There are two other very important aspects of~\eqref{eq:tym-brst}. First, its Witten index vanishes --- hinting the possibility that the BRST symmetry might be unstable~\cite{witten1982a,fujikawa1983a}. This is an indication that a non-topological phase can emerge out of TYM\@. Second, its cohomology groups forbid the presence of $ \mathrm{g} $ metric-contaminated observables, including the traditional YM Lagrangian density $ \tr \left( F \star F \right) $\footnote{Implicitly $\mathrm{g}$ metric-contaminated due to the explicit presence of $\star$.}. If the BRST turns out stable, then all non-trivial local observables are elements in the $s$-cohomology modulo $d$-boundaries. In this case, the only allowed are
\begin{equation}\label{eq:tym-observables} \mathcal{O}_n = \tr \left( \tilde{F}^n \right) \; ; \; n \in {\mathbb{N}}^{\ge 1} \;. \end{equation} These are invariant polynomials of $\tilde{F}$ which give Chern classes of the universal bundle. In other words, they are all topological in nature. In particular, \begin{equation}\label{eq:donaldson-polynomials} \mathcal{O}_2 = \tr \left[ F^2 + 2\psi F + \left( 2\phi F + \psi^2 \right) + 2\psi \phi + \phi^2\right]
\end{equation}
contains, precisely, the Donaldson polynomials first evaluated in the seminal works of S.~K.~Donaldson~\cite{donaldson1983a,donaldson1990a}, and E.~Witten~\cite{witten1988d} --- albeit for the case $N=2$\footnote{$N>2$ represent generalizations of the original polynomials, reflecting the slightly more involved nature of $SU(N>2)$ bundles over $\mathbb{R}^4_{\infty}$.}.

Among all allowed observables in~\eqref{eq:tym-observables}, only the (4,0) component of~\eqref{eq:donaldson-polynomials} is suitable for a Lagrangian density in dimension 4. Thus, the TYM symmetries force the action functional to be
\begin{equation}%
  \label{eq:tym-action}
  S_{\text{TYM}}\left[ A \right]  \equiv  \int \tr\left(F^2\right)
  = 8 \pi^2 k \;,
\end{equation}
where $ k \in \mathbb{Z} $ is the \textit{instanton number}\footnote{The instanton number is generally non-vanishing on $\mathbb{R}^{4}_{\infty}$. It is well-known that $F^2$ is a global $d$-cycle. Poincaré lemma then implies $F^2$ is a local $d$-boundary. In fact, $F^2 = d \omega^{(3)} $, where $\omega^{(3)} \equiv AdA+\tfrac{2}{3}A^3$ is the famous Chern-Simons 3-form. Nonetheless, the 4th de Rham cohomology group of $\mathbb{R}^{4}_{\infty}$, namely $ H^4_{\text{dR}} \left( \mathbb{R}^{4}_{\infty}\right) \sim H^4_{\text{dR}} \left( S^{4}\right) \sim \mathbb{Z} $, obstructs $F^2$ to be a global $d$-boundary, \textit{i.e.}, under an integral sign $F^2$ does not equate $d \omega^{(3)}$.} of $A$. More precisely, $k$ is the 2nd Chern number of the $SU(N)$ bundle over $\mathbb{R}^{4}_{\infty}$, and on which the principal connection associated with $A$ was originally defined. This is an important distinction because $k$ classifies these bundles. Gauge fields (modulo gauge transformations) are in one-to-one correspondence with them (modulo bundle isomorphisms). In this sense, TYM is explicitly a topological field theory, and~\eqref{eq:tym-action} is explicitly a smooth invariant.

Unsurprisingly, the field equations are trivial ($0=0$), signaling lack of local bulk dynamics. Many topological field theories can be formulated as fully extended functorial field theories~\cite{atiyah1988a,segal1988a,baez1995a,schreiber2009a,baez2009a}. In this context, their non-local bulk (or boundary) dynamics can be roughly understood as the propagation and scattering of smoothly embedded fully extended cobordisms. Moreover, TYM has vanishing (pure gauge) energy functional on the bulk, but not necessary on the celestial sphere at infinity
\begin{equation}%
  \label{eq:tym-energy}
  \lim_{ r \to \infty}  \mathbb{E}_{\text{TYM}} [A; r] = \int_{S^{2}_{\infty}} \tr \left[2\left(\lambda \rfloor A\right)F\right] \;,
\end{equation}
where $\lambda$ is the globally constant vector field generating spacetime translations, and $S^{2}_r \equiv \partial B^3_r$ is the 2-sphere boundary of the 3-ball $B^3_r \subset \mathbb{R}^{4}_{\infty} $ of radius $r$. In this sense, every bulk field configuration amounts to the gauge vacua.

A partition function for TYM, formally defined in the weakly coupled regime, requires~\eqref{eq:tym-action} to be gauge fixed. Quantum TYM (QTYM) has been studies in several different gauge choices~\cite{baulieu1988a,myers1990c,brandhuber1994a,piguet1995a,sadovski2017c,sadovski2018a,sadovski2018b}. A particularly convenient one is the (anti-)self-dual Landau ((A)SDL) conditions
\begin{subequations}\label{eq:asdlg}
  \begin{align}
    d \star A    & = 0 \;, \\
    d \star \psi & = 0 \;, \\
    F^\pm        & = 0 \,,
  \end{align}
\end{subequations}
where $ F^\pm \equiv F \pm \star F $. This choice localizes our partition function on the very well-behaved $SU(N)$ (anti-)instanton moduli, where the Donaldson invariants were originally formulated via intersection theory. The (A)SDL gauge results in a very strong set of Ward identities. QTYM is shown to be renormalizable to all orders in perturbation theory with only one independent, and non-physical, renormalization~\cite{sadovski2017c}. Moreover, all connected $n$-point Green functions are tree-level exact~\cite{sadovski2018a}. Clearly, this gauge makes evident that QTYM also has no local degrees of freedom. And, due to the lack of loop corrections, the classical observables in~\eqref{eq:tym-observables} maintain their topological nature: the QTYM local observables are still the Donaldson polynomials. It is worth mentioning that the $ s $-cohomology also guarantees that QTYM is free of gauge anomalies~\cite{baulieu1988a}.

Finally, we comment that \textit{Gribov ambiguities} are necessarily present in QTYM due to the non-triviality of its gauge moduli bundle, $\mathcal{G} \hookrightarrow \mathcal{A} \rightarrow \mathcal{M}$~\cite{gribov1978a,singer1978a}. However, a refinement of the gauge fixing procedure cannot yield any interesting new physics because QTYM is tree-level exact. In this sense, we say its Gribov problem is trivial~\cite{sadovski2020a}.
\end{document}
