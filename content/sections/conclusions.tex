\documentclass[../main.tex]{subfiles}

\begin{document}

\section{Conclusions}%
\label{sec:conclusions}

Our analysis have demonstrated that four-dimensional $SU(N)$ invariant TYM theories, when minimally coupled to Fujikawa-Higgs fields, accept representation in terms of massive gauge fields in a non-trivial neighborhood of their moduli. This is a general feature when all fields are in the adjoint representation (aTYMH theories), while it relies on the realification $SU(N) \hookrightarrow SO(2N)$ when the Fujikawa-Higgs fields are in the fundamental representation of the gauge group (TYMH$_{\mathbb{R}}$ theories). In particular, this is a surprising result in the context of topological field theories since they are supposed to be scale invariant theories.

The emergence of mass scales may be physically insignificant if path integrals remain unaffected by said parameters. This would be a scenario similar to how cohomological TQFTs retain scale invariance even when faced with non-vanishing $\beta$-functions of their coupling parameters. Most notably, Witten's $SU(2)$ $\mathcal{N}=2$ twisted Super-Yang-Mills theory~\cite{witten1988d,brooks1988a}. However, for such a scale invariance to be realized, the quantum theories need very strong sets of Ward identities. Decisively, the Slavnov-Taylor identities inherited from topological BRST symmetries. Saying it differently, the theories need topological BRST stability --- precisely what aTYMH and TYMH$_{\mathbb{R}}$ lack in their Higgs phases.

We have shown that the BRST instability of aTYMH theories happens beyond tree-level, and closely resembles the Coleman-Weinberg mechanism if loop corrections are to blame. Accordingly, the emergence of  $N^2 - \sum_l n_l^2$ massive gauge fields only carries physical weight in the quantum realm. Similarly, the tree-level Higgs-phase energy functional $\bar{\mathbb{E}}_{\text{aTYMH}}$ is bulk trivial, but $\langle 0 \rvert \bar{\mathbb{E}}_{\text{aTYMH}} \lvert 0 \rangle$ is not --- hinting a quantum localization of the effective gauge vacua onto the moduli of flat connections. As collateral result, $m-1$ independent topological solitons of spatial co-dimension 3, associated with the spontaneous gauge symmetry breaking $\mathfrak{su}(N) \mapsto \bigoplus_l \mathfrak{su}(n_l) \oplus \bigoplus^{m-1} \mathfrak{u}(1)$, are quantum mechanically stable. We conclude that Higgs-phase quantum aTYMH theories are generically populated with massless and massive gauge field states, Higgs field states, Nambu-Goldstone fermionic field states, as well as vertex or monopole non-local states depending on whether the metric structure of spacetime is Riemannian or Lorentzian, respectively.

Unexpectedly, the BRST instability of TYMH$_{\mathbb{R}}$ theories more pronouncedly happens at tree-level. We can explicitly observe the emergence of a Nambu-Goldstone fermion in the effective action~\eqref{eq:real-eff-action}. We can also observe $4N-3$ massive gauge field components, associated with the realified spontaneous gauge symmetry breaking $\mathfrak{so}(2N) \mapsto \mathfrak{so}(2N-2)$, implying the emergence of a physical mass scale at classical level. Similarly, the tree-level Higgs-phase energy functional $\bar{\mathbb{E}}_{\text{TYMH}_{\mathbb{R}}}$ is bulk non-trivial --- hinting a classical localization of the effective gauge vacua on the moduli of flat connections. Stable solitonic degrees of freedom are, however, quite rare. Only $N=2$ produces a single topological soliton of spatial co-dimension 3. Upon quantization, the Higgs-phase effective TYMH$_{\mathbb{R}}$ theories are generically populated with massless and massive gauge field states, Higgs field states, and Nambu-Goldstone fermionic field states.

Ultimately, topological BRST instability implies that local degrees of freedom are generically present in the Higgs phases of these topological field theories. We speculate that the adjoint case might be an interesting realization of the Coleman-Weinberg mechanism~\cite{coleman1973a}, while the realified fundamental case might bridge high energy topological gravity models to low energy geometrodynamics~\cite{sako1997a,mielke2011a,alexander2016a,sadovski2017a,chengzheng2017a,agrawal2020a,edery2023a,tianyao2023a,kehagias2021a,raitio2024a,sadovski2025a}.
\end{document}
