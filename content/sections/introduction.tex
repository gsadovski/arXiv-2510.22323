\documentclass[../main.tex]{subfiles}
\begin{document}

\section{Introduction}%
\label{sec:introduction}

Dynamical BRST\footnote{Becchi-Rouet-Stora-Tyutin.} symmetry breaking is an exceptionally rare phenomenon in quantum field theory. This is largely because the BRST charge $Q$ nilpotency, and its overall cohomological structure, is protected by a topological invariant of the field space --- the Witten index $\tr {(-1)}^F$~\cite{witten1981a,witten1982a,fujikawa1983a}. A non-vanishing index guarantees the balance between bosonic and fermionic zero modes, ensuring that the BRST symmetry remains unbroken. Physical observables can then be consistently defined in the classical and quantum realm through its cohomology groups. Not coincidently, most consistent BRST invariant quantum field theories are defined in a regime where $Q \lvert 0 \rangle = 0$.

Historically, any sign of BRST symmetry breaking was regarded as a pathology --- an indication that the gauge fixing and/or quantization procedure had failed to properly isolate the subspace of physical states. However, the modern understanding has reshaped this perspective. In the Gribov–Zwanziger framework, for instance, the BRST instability --- made explicit by the restriction of gauge field configurations to the Gribov region ---, is interpreted as a possible hint of color confinement in non-Abelian gauge theories~\cite{gribov1978a,zwanziger1989a,maggiore1994a,sorella2009a,dudal2012a,vandersickel2012a}\footnote{Recent developments have shown that the breaking is not catastrophic, and Gribov-Zwanzinger theory can be recast in a new BRST invariant form in several different choices of gauge functions~\cite{pereira2015a,capri2015a,pereira2016a,pereira2016b,pereira2016c}.}. Similarly, in the Parisi–Sourlas framework of stochastic dynamics, unstable BRST symmetries can be related to the emergence of chaos, long-time correlations, $1/f$ noise, \textit{etc.}~\cite{parisi1982a,ovchinnikov2016a,ovchinnikov2025a}. What was once considered a breakdown in consistency, is now understood as a window into non-perturbative phenomena.

In this work, we analyze spontaneous BRST and gauge symmetry breaking in a class of topological quantum field theories (TQFTs) known as \textquote{topological Yang-Mills} (TYM)\footnote{There is no standard use of the term \textquote{TYM} in the literature. It can mean field theories whose Lagrangian density is $\tr (F \wedge F)$, also referred to as Baulieu-Singer theories~\cite{baulieu1988a}. But, it can also mean twisted $\mathcal{N}=2$ Super-Yang-Mills theories, also referred to as Donaldson-Witten theories~\cite{witten1988d,witten1991a}. We adopt the former meaning (a comparison between the two can be found in~\cite{junqueira2021a}).}. As we briefly review in Section~\ref{sec:tym}, these are $SU(N)$ gauge invariant topological field theories, whose BRST quantization give exact path integral representations for the so-called Donaldson polynomials --- smooth invariants of 4-manifolds, carrying global information about differentiable structures~\cite{donaldson1983a,donaldson1990a,donaldson1990b,witten1988d,baulieu1988a,witten1991a}. Since the most current experimental and observational data support the existence of four large physical dimensions~\cite{lee2020a,wenhai2020a,chakravarti2020a,visinelli2018a,vagnozzi2019a}, TYM theories acquire immediate relevance in our understanding of gravity, and the large-scale structure of spacetime~\cite{witten1988a,brans1992,maluga1996,maluga2007}.

TYM theories, remarkably, also have relevance in the non-perturbative sector of quantum chromodynamics (QCD). This is because Donaldson polynomials are intrinsically related to intersection forms on the Yang-Mills (YM) instanton moduli. These are precisely the non-perturbative field configurations responsible for the huge degeneracy of the QCD vacuum. In fact, the induced non-vanishing vacuum-to-vacuum amplitudes can change the chirality of fermions, explaining the $U(1)$ axial anomaly and, consequentially, the huge mass of the $\eta '$ meson~\cite{thooft1976a,thooft1986a}. Clearly, results in TYM theories can, potentially, trickle down to phenomenological implications in particle physics just as well.

TYM theories are cohomological field theories, also known as Witten-type TQFTs. This means their defining symmetry is a \textquote{topological BRST symmetry} --- their BRST operator is the differential for a model of equivariant cohomology of a field space. On one hand, this is precisely what guarantees their subspaces of physical states to be populated exclusively by BRST invariant vacuum states (no local excitations), and their set of observables to be exclusively topological in nature. On the other hand, it greatly hinders any naïve attempts to directly connect TYM theories with traditional local physics.

Interestingly enough, TYM theories have vanishing Witten index --- its topological BRST symmetry is unprotected, possibly unstable. This is precisely the direction we focus our investigation. We find that, when adequately coupled to \textquote{Higgs-like fields}, these theories do exhibit BRST instability, revealing non-topological Higgs-like phases. In particular, the tree-level emergence of local degrees of freedom, and of a physical mass scale, due to \textquote{spontaneous topological BRST symmetry breaking} --- found in the realified $SU(N) \hookrightarrow SO(2N)$ theories (see Section~\ref{sec:fundamental}) --- might be especially interesting to connect high energy topological gravity models to low energy geometrodynamics~\cite{sako1997a,mielke2011a,alexander2016a,sadovski2017a,chengzheng2017a,agrawal2020a,edery2023a,tianyao2023a,kehagias2021a,raitio2024a,sadovski2025a}.

This work is structured as follows. In Section~\ref{sec:tym}, we define our conventions, establish the mathematical framework, and review the main introductory features of TYM theories. In Section~\ref{sec:adjoint}, we develop the so-called \textquote{Fujikawa-Higgs sectors} of these theories, with all fields in the adjoint representation of $SU(N)$. We evaluate their Higgs phase effective theories, and argue for the BRST instability beyond tree-level --- implying these exhibit non-trivial quantum mechanical regimes. In particular, these regimes must be generically populated with local degrees of freedom such as massless and massive gauge field states, and Nambu-Goldstone fermionic field states. There is, in addition, the widespread presence of non-local field states associated with stable topological solitons. In Section~\ref{sec:fundamental}, we follow very closely the methodology employed in Section~\ref{sec:adjoint}, but with the \textquote{Fujikawa-Higgs fields} in the fundamental representation of $SU(N)$. We very quickly find that no spontaneous symmetry breaking of any kind occurs, unless we appeal to realification. We proceed by developing the realified $SU(N) \hookrightarrow SO(2N)$ theories, evaluating their Higgs phase effective theories, and arguing for the tree-level BRST instability. In particular, the classical regimes already exhibit the presence of local degrees of freedom such as massless and massive gauge fields, and Nambu-Goldstone fermionic fields. Albeit possible at a classical level, almost no topological solitons are generated. Finally, Section~\ref{sec:conclusions} contains our conclusions.

\end{document}
