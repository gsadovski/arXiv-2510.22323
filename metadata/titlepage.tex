\documentclass[main.tex]{subfiles}

\begin{document}
\thispagestyle{empty}
%%%%%%%%%
% TITLE %
%%%%%%%%%
{
  \raggedright % chktex 1
  % \noindent
  \Large
  \bfseries
  Four-dimensional topological Yang-Mills-Higgs theories with BRST instability\par % chktex 36
  \bigskip
  \bigskip
}

%%%%%%%%%%%
% AUTHORS %
%%%%%%%%%%%
\begin{addmargin}[6mm]{0mm}
  {
    \noindent
    \bfseries
    Guilherme~Sadovski
    \bigskip
  }

  %%%%%%%%%%
  % EMAILS %
  %%%%%%%%%%
  {
    \noindent
    \footnotesize
    \rmfamily
    e-mail: \href{mailto:gsadovski@proton.me}{gsadovski@proton.me}
    \bigskip
  }

  %%%%%%%%%%%%
  % ABSTRACT %
  %%%%%%%%%%%%
  {
    \noindent
    \bfseries
    Abstract.\normalfont{} We show that four-dimensional topological Yang-Mills theories, when suitably coupled to Higgs-like fields, admit representations in terms of massive gauge fields in a non-trivial neighborhood of the minima moduli. In the adjoint representation, a beyond tree-level BRST instability is present, closely resembling the Coleman-Weinberg mechanism. The fundamental representation requires realification $SU(N) \hookrightarrow SO(2N)$, but exhibit a pronounced tree-level instability. Stable solitons (vertices/monopoles) are generically present in the adjoint case. These BRST instabilities indicate the reintroduction of local degrees of freedom into the physical spectrum of these theories. In particular, the realified fundamental case may provide a promising framework for 4d topological gravity models. In addition, our results offer rare examples of BRST symmetry breaking in non-Abelian gauge theories with trivial Gribov problem.
  }
\end{addmargin}
\end{document}
