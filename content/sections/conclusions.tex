\documentclass[../main.tex]{subfiles}

\begin{document}

\section{Conclusions}%
\label{sec:conclusions}

Our analysis have demonstrated that four-dimensional $SU(N)$ invariant TYM theories, when adequately coupled to Higgs-like fields, do accept representation in terms of massive gauge fields in a non-trivial neighborhood of their moduli. This is a general feature when all fields are in the adjoint representation (which we named aTYMH theories), while it relies on the realification $SU(N) \hookrightarrow SO(2N)$ of our structure group, given the bosonic potential at hand, when the Higgs-like fields are in the fundamental representation (which we named TYMH$_{\mathbb{R}}$ theories). This is a surprising result in the context of topological field theories since they are supposed to be scale invariant.

It could be argued that the emergence of a mass scale is harmless, as long as their path integrals remain independent of such parameters. This is precisely what happens to pure QTYM and, more generally, to TQFTs of the cohomological type. However, for this scenario to be realized, the quantum theory needs a very strong set of Ward identities. In particular, the presence of a Slavnov-Taylor identity associated with \textquote{a topological BRST symmetry.} This is very much not the case with aTYMH and TYMH$_\mathbb{R}$, as we have seen that topological BRST instability is a general feature in the Higgs phase of these theories.

BRST instability in aTYMH theories happens beyond tree-level, and closely resembles the Coleman-Weinberg mechanism if loop corrections are to blame. Further investigation is needed to determine its quantum mechanical origin. However, it certainly generates a bulk non-trivial effective energy functional, $\langle \bar{\mathbb{E}}_{\text{aTYMH}}\rangle \neq 0$, hinting a localization of the effective gauge vacua on the moduli of flat connections. As a collateral result, it stabilizes solitonic degrees of freedom associated with the gauge symmetry breaking. The Higgs phase of aTYMH theories is populated with vertices or monopoles, depending if the spacetime metric structure is Riemannian or Lorentzian, respectively.

BRST instability is also a general feature in the Higgs phase of these theories. In the realified fundamental representation it happens already at tree-level, while in the adjoint one it happens beyond tree-level, and it is potentially related to the Coleman-Weinberg mechanism.

but usually necessitates a very strong set of Ward identities. In a cohomological field theory such as QTYM, this role is played by Slavnov-Taylor identity inherited from the topological BRST symmetry.
\end{document}
