\documentclass[../main.tex]{subfiles}

\begin{document}

\section{Topological Yang-Mills symmetries}\label{sec:tym}

Let spacetime be $\mathbb{R}^{4}_{\infty}$, endowed with a globally flat metric $\mathrm{g}$. The 4-manifold $\mathbb{R}^{4}_{\infty}$ is the one-point compactification of the standard $\mathbb{R}^4$, compactible with its conformal structure at infinity. This is a convenient choice of background for TYM, because it is compact, Riemannian and 4-dimensional --- necessary conditions for the rigorous definition of the YM instanton\footnote{$\mathbb{R}^{4}_{\infty}$ is also homeomorphic, but not diffeomorphic, to $S^4$. Additionally, $\mathrm{g}$ is conformally equivalent to a round metrics on the 4-sphere. Thus, CFTs on $\left( \mathbb{R}^{4}_{\infty}, \mathrm{g}\right)$ can conveniently pick up non-trivial topological effects, though they are defined on a flat background.}. Due to the nature of topological field theories, we will work with differential forms. For future reference, $\wedge$ is the exterior product\footnote{The tensor product symbol $\otimes$, its fully symmetrized version $\vee$, and its fully anti-symmetrized version $\wedge$, will be omitted whenever the context is sufficiently clear.}, $d$ is the exterior derivative, $v \rfloor$ is the interior product along the smooth vector field $v$, and $\star$ is the Hodge dual. All these operations occur on the exterior algebra $\bigwedge \mathbb{R}^{4}_{\infty}$ of spacetime.

Let $SU(N)$ with Lie algebra $\mathfrak{su}(N)$ be our structure group. This is a physically relevant choice, but also mathematically convenient as $SU(N)$ YM instantons are reasonably well-understood\footnote{The moduli spaces of irreducible self-dual $SU(N)$ connections are finite-dimensional, generically smooth and Kähler. Additionally, $SU(N)$ instantons can be fully classified via the ADHM construction.}. Also for future reference, we define on each event $x \in \mathbb{R}^{4}_{\infty}$ the $\mathbb{Z}_2$ graded\footnote{The grading $|X|$ of $X$ will be the sum mod 2 of its differential form rank and ghost number.} Lie bracket $[X,Y] \equiv XY-(-1)^{|X||Y|}YX$, the adjoint representation $\mathrm{ad}_X (Y) \equiv \left[X, Y\right]$, and the Killing form $\kappa \left(X,Y\right) \equiv \tr \left( \mathrm{ad}_{X} \circ \mathrm{ad}_{Y} \right)$ acting on $X, Y \in \mathfrak{su}(N) \otimes \bigwedge_x \mathbb{R}^{4}_{\infty}$.

The full geometrical arena of a $SU(N)$ gauge theory over $\mathbb{R}^{4}_{\infty}$ is that of the universal bundle $\left( SU(N)\times\mathcal{G} \right) \hookrightarrow \left( P \times \mathcal{A} \right) \rightarrow \left( \mathbb{R}^{4}_{\infty} \times\mathcal{A}/\mathcal{G} \right)$~\cite{baulieu1985a,baulieu1988a}. The smooth  $\left(N^2+3\right)$-manifold $P$ is part of the more commonly known principle bundle structure $SU(N) \hookrightarrow P \rightarrow \mathbb{R}^{4}_{\infty}$, together with its adjoint bundles, $ \mathrm{Ad} P \equiv P \times_{SU(N)} SU(N) $ and $ \mathrm{ad}P \equiv P \times_{SU(N)} \mathfrak{su}(N)$. The space of all smooth $SU(N)$-connections on $ P $ is $ \mathcal{A} \equiv C^{ \infty } \left( J^1 P \right) $, where $ J^1 P $ is its 1st jet bundle. The space of all smooth gauge transformations is $ \mathcal{G} \equiv C^{ \infty }\left( \mathrm{Ad}P \right) $, and $ C^{ \infty } \left( \mathrm{ad} P \right) $ is its Lie algebra.

A gauge field $A(x)$ is traditionally defined as an element of $C^\infty \left( \mathrm{ad}P \otimes \bigwedge^1\mathbb{R}^{4}_{\infty} \right)$ --- an $ \mathrm{ad} P $-valued 1-form field on $\mathbb{R}^{4}_{\infty}$. And, it results from the pullback to spacetime of an $SU(N)$-connection living on $P$. Similarly, an universal gauge field $\tilde{A}(x)$ is the result of a pullback to $\mathbb{R}^{4}_{\infty} \times \mathcal{A}/\mathcal{G}$ of a universal $\left( SU(N) \times \mathcal{G} \right)$-connection living on $P \times \mathcal{A}$. The universal gauge field can be written as
\begin{equation}
  \tilde{A} = A + c \;,
\end{equation}
where $c(x) \in C^\infty \left( \mathrm{ad}P \otimes \bigwedge^0\mathbb{R}^{4}_{\infty} \right)$ is an $\mathrm{ad}P$-valued 0-form which results from a local projection of the Maurer-Cartan form on $ \mathcal{G} $ --- the well-known Faddeev-Popov ghost field. We can roughly understand $A$ and $c$ as the (1,0) and (0,1) component of $\tilde{A}$, respectively, in relation to the product $\mathbb{R}^{4}_{\infty} \times \mathcal{A}/\mathcal{G}$.

The exterior derivative $\tilde{d}$ on $\mathbb{R}^4 \times \mathcal{A}/\mathcal{G}$ can be written as
\begin{equation}%
  \label{eq:universal-exterior-derivative}
  \tilde{d} = d + s \;,
\end{equation}
where $s$ is the exterior derivative on $\mathcal{G}$ --- the well-known BRST operator. The $\mathbb{Z}_2$ graded exterior algebra we already mentioned is defined by $\tilde{d}$, which gives the meaning of an 1-form with ghost number 0, and a 0-form with ghost number 1, respectively, to $ \left( 1,0 \right) $, and $ \left( 0,1 \right) $. Accordingly, the grading (statistics) of $A$, and $c$, is odd (fermionic) --- the grading of all fields can be found at Table~\ref{tab:gradings}. Additionally, $\tilde{d}$, $d$, and $s$ are all nilpotent operators by definition. Equation~\eqref{eq:universal-exterior-derivative} resumes to $sd+ds=0$, which means $d$ and $s$ are fermionic operators.

The universal curvature $ \tilde{F} $ of $\tilde{A}$ is given by
\begin{subequations}%
  \label{eq:universal-curvature}
  \begin{align}
    \tilde{F} & \equiv \tilde{d}\tilde{A}+\tilde{A}^2\;, \\
              & = F+\psi+\phi \;,
  \end{align}
\end{subequations}
and it has the curvature $F\equiv dA+A^2$ of $A$ as the (2,0) component, the so-called \emph{topological ghost field} $\psi \equiv sA + Dc$ as the (1,1) component, and the \emph{2nd generation topological ghost field} $\phi \equiv sc + c^2$ as (0,2) component. We have to mention that $D \equiv d+ \mathrm{ad}_A$ is the exterior covariant derivative in the adjoint representation.

The Bianchi identities for $\tilde{F}$, and $F$, give
\begin{subequations}
  \label{eq:universal-bianchi-identity}
  \begin{align}
    \tilde{D}\tilde{F}          & =0 \;,  \\
    sF+D\psi+\left[ c,F \right] & = 0 \;.
  \end{align}
\end{subequations}
And, from them, we can finally read the so-called TYM BRST symmetry transformations~\cite{baulieu1988a},
\begin{subequations}\label{eq:tym-brst}
  \begin{align}
    sA    & = -Dc + \psi \;,                      \\
    sc    & = - c^2 + \phi \;,                    \\
    s\psi & = -D\phi - \left[ c, \psi \right] \;, \\
    s\phi & = - \left[ c, \phi \right]\;,         \\
    sF    & = -D\psi - \left[ c, F \right] \;.
  \end{align}
\end{subequations}
This is the most general realization of the BRST algebra, capturing the full mathematical structure of traditional gauge theories. It also represent a very strong set of symmetries. In contrast, the much weaker YM BRST of particle physics can be obtained by imposing the so-called \textit{horizontal condition}, $\psi=\phi=0$.

The $s$-cohomology forbids the presence of the traditional YM Lagrangian density, $ \tr \left( F \star F \right) $. In fact, it forbids the presence of any $ \mathrm{g} $~metric-contaminated observable. In the absence of spontaneous BRST symmetry breaking, all non-trivial observables are elements in the $s$-cohomology modulo $d$-boundaries. The only non-trivial ones allowed by~\eqref{eq:tym-brst} are of the type
\begin{equation}\label{eq:tym-observables}
  \mathcal{O}_k = \tr \left( \tilde{F}^k \right) \; ; \; k \in {\mathbb{N}}^{\ge 1} \;.
\end{equation}
These invariant polynomials of $\tilde{F}$ can be used to construct the Chern classes of the universal bundle. In other words, they are all topological in nature. In particular,
\begin{equation}\label{eq:donaldson-polynomials}
  \mathcal{O}_2 = \tr \left[ F^2 + 2\psi F + \left( 2\phi F + \psi^2 \right) + 2\psi \phi + \phi^2\right]
\end{equation}
contains, precisely, the Donaldson polynomials evaluated in the seminal works of S.~K.~Donaldson~\cite{donaldson1983a,donaldson1990a}, and E.~Witten~\cite{witten1988d}.

Among all allowed observables in~\eqref{eq:tym-observables}, only the (4,0) component of~\eqref{eq:donaldson-polynomials} is suitable for a Lagrangian density at four spacetime dimensions. Thus, the TYM action functional is defined to be
\begin{equation}\label{eq:tym-action}
  S_{\text{TYM}}\left[ A \right] \equiv  \int\tr\left( g F^2\right) \;,
\end{equation}
where $ g $ is a dimensionless coupling parameter. The Lagrangian density $\tr \left(g F^2 \right)$ is proportional to the 1st Pontryagin number of $\mathbb{R}^4$. And, its integral is proportional to the (compactly supported) Hirzebruch signature of $\mathbb{R}^4$. In other words,~\eqref{eq:tym-action} is a topological  invariant of spacetime.

The field equations are trivial ($0=0$), which signals a lack of local dynamics. Instead, TYM has a non-trivial non-local dynamics in the bulk. Many topological field theories can be formulated as fully extended functorial field theories~\cite{atiyah1988a,segal1988a,baez1995a,schreiber2009a,baez2009a}. In this context, their non-local bulk dynamics can be roughly understood as the propagation and scattering of topologically embedded fully extended cobordisms.

\subsection{Quantum properties}\label{ssec:quantum-properties;sec:tym}

A partition function for TYM, formally defined in the weakly coupled regime, requires~\eqref{eq:tym-action} to be gauge fixed. Quantum TYM (QTYM) has been studies in several different gauge choices~\cite{baulieu1988a,myers1990c,brandhuber1994a,piguet1995a,sadovski2017c,sadovski2018a,sadovski2018b}. Here, we adopt the (anti-){}self-dual Landau ((A){}SDL) conditions
\begin{subequations}\label{eq:asdlg}
  \begin{align}
    d \star A    & = 0 \;, \\
    d \star \psi & = 0 \;, \\
    F^{\pm}      & = 0 \,,
  \end{align}
\end{subequations}
where $ F^{\pm} \equiv F \pm \star F $.

The $ s $-cohomology guarantees that QTYM is free of gauge anomalies~\cite{baulieu1988a}. The (A){}SDL gauge choice is convenient because it results in a very strong set of Ward identities. In this gauge, QTYM is shown to be renormalizable to all orders in perturbation theory. It has only one independent, and non-physical, renormalization~\cite{sadovski2017c}. Moreover, all connected $n$-point Green functions are tree-level exact~\cite{sadovski2018a}. Clearly, this gauge makes evident that QTYM also has no local degrees of freedom. And, due to the lack of loop corrections, the classical observables in~\eqref{eq:tym-observables} maintain their topological nature: the QTYM observables are still Donaldson polynomials.

\begin{table}[htpb]
  \caption{Grading of all TYM-FH fields.}%
  \label{tab:gradings}
  \begin{tabular}{cccccccccccc}
    \toprule
    Field      & $A$ & $F$  & $c$ & $\psi$ & $\phi$ & $\Phi$ & $\xi$ & $\eta$ & $B$  \\
    \midrule
    Form rank  & 1   & 2    & 0   & 1      & 0      & 0      & 0     & 0      & 0    \\
    Ghost no.  & 0   & 0    & 1   & 1      & 2      & 0      & 1     & -1     & 0    \\
    Statistics & odd & even & odd & even   & even   & even   & odd   & odd    & even \\
    \bottomrule
  \end{tabular}
\end{table}
\end{document}
