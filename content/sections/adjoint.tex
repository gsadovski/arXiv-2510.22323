\documentclass[../main.tex]{subfiles}

\begin{document}

\section{Adjoint TYMH theories}%
\label{sec:adjoint}

We start our investigation on the spontaneous BRST and gauge symmetry breaking by recalling that the traditional Higgs action functional is $\mathrm{g}$ metric-contaminated, and not an $s$-boundary. Clearly, this is not an ideal candidate to couple to TYM theories as it would explicitly break their topological symmetries. Simultaneously, a naïve introduction of a scalar fields can give non-vanishing contributions to the Witten index of the surviving BRST symmetry, jeopardizing spontaneous instability scenarios.

To circumvent these issues, we build upon the Higgs-like toy model first proposed by K.~Fujikawa in~\cite{fujikawa1983a}. Fujikawa's original work is not gauge invariant, so what follows is our gauge invariant generalizations of it, first in the adjoint, and later in the fundamental (and realified fundamental) representations of $SU(N)$. Albeit we focus on the coupling to TYM theories, the methodology below is actually adaptable to many other topological gauge field theories regardless of the spacetime dimension they are defined.

Consider the field variables $\Phi (x)$, $\eta (x)$, $\xi (x)$, $B (x)$, in $C^{\infty} ( \mathrm{ad} P \otimes \bigwedge^0 \mathbb{R}_{\infty}^{4} )$, and obeying the algebra
\begin{subequations}%
  \label{eq:afh-brst}
  \begin{align}
    s \Phi & = - \left[c, \Phi\right] + \eta \;, \\
    s \eta & = -\left[c, \eta\right]  \;,        \\
    s \xi  & = - \left[c, \xi\right] + B \;,     \\
    s B    & = -\left[c, B\right]  \;.
  \end{align}
\end{subequations}
We will refer to them as adjoint Fujikawa-Higgs (aFH) fields, and their statistics can be found at Table~\ref{tab:gradings}. Introduced as a pair of $s$-doublets, aFH fields can only build gauge invariant polynomials which are $s$-boundaries. This is a general result in BRST algebra, known as the \textit{doublet theorem}. It justifies the form of the action functional~\eqref{eq:afh-action-s-exact}, also implying that the $s$-cohomology groups are explicitly preserved: no new observables are introduced, especially no $\mathrm{g}$ metric-contaminated ones. In other words, the topological symmetries are left untouched. Additionally, $s$-doublets generically give vanishing contributions to the Witten index of $s$. In this way, their introduction cannot naïvely change the BRST (in)stability scenario. % chktex 36

We choose to minimally couple TYM and aFH fields, defining the Fujikawa-Higgs sector of the $SU(N)$ invariant \textquote{adjoint topological Yang-Mills-Higgs} (aTYMH) theories via the action functional
\begin{subequations}%
  \label{eq:afh-action}
  \begin{align}%
    % S_{\text{aFH}} \left[A, \psi, \Phi, \xi, \eta, B\right] & \equiv s \int \tr \left\{D \xi \star D \Phi + \xi \Phi \left[m^2 + g \left(B \Phi - \xi \eta\right)\right] \star \mathds{1} \right\} \;,                      \label{eq:afh-action-s-exact} \\
    S_{\text{aFH}} \left[ \Xi \right] & \equiv s \int \tr \left\{D \xi \star D \Phi + \xi \Phi \left[m^2 + g \left(B \Phi - \xi \eta\right)\right] \star \mathds{1} \right\} \;,                      \label{eq:afh-action-s-exact} \\
                                      & = \int \tr \left\{ -DB \star D \Phi - D \xi \star D \eta + \left( \mathrm{ad}_{\Phi}D \xi - \mathrm{ad}_{\xi} D \Phi \right) \star \psi \vphantom{m^2} \right. + \nonumber                  \\
                                      & + \left. \left[ m^2 \left( B \Phi - \xi \eta \right) + g {\left( B \Phi - \xi \eta \right)}^{2} \right] \star \mathds{1}  \right\} \;, \label{eq:afh-action-full}
  \end{align}
\end{subequations}
where $\Xi$ is shorthand for $A, \psi, \Phi, \eta, \xi, B$; $m^2$ and $g$ are mass and coupling parameters, respectively, and; $\star \mathds{1}$ is the unitary volume 4-form on spacetime\footnote{This functional is, in fact, agnostic about the spacetime dimension.}. This is not the most general minimally coupled $s$-boundary we could have built out of aFH fields. The addition of more terms, \textit{e.g.}, $s \tr (D \xi \star DB)$, most likely benefits its renormalizability properties, but does not alter its symmetry breaking patterns. Additionally, we have omitted all $c$-vertices from~\eqref{eq:afh-action-full}, which is equivalent to work with the already mentioned Cartan model --- isomorphic to the Kalkman model in the subspace of basic forms. It can be shown that the integrand in~\eqref{eq:afh-action-s-exact} is indeed a basic form.

For convenience, we will work with the Cartan model whenever we are dealing with basic forms. Explicitly,
\begin{subequations}%
  \label{eq:cartan-model-afh-brst}
  \begin{align}
    s A    & =   \psi \;,   & s \Phi & =   \eta \;, & \hspace{200pt} \\
    s \psi & =  - D\phi \;, & s \eta & = 0 \;,      & \hspace{200pt} \\
    s \phi & = 0 \;,        & s \xi  & =  B  \;,    & \hspace{200pt} \\
    s F    & = -D\psi       & s B    & = 0      \;, & \hspace{200pt}
  \end{align}
\end{subequations}
and, now, $s$ is nilpotent only up to infinitesimal gauge transformations, $\delta_{\phi} \equiv - s^2$, generated by $\phi$. The integrand~\eqref{eq:afh-action-s-exact} is basic precisely because it is independent of $\phi$, and invariant under $\delta_{\phi}$. The aFH fields also individually fit this definition, further implying they all commute with $\phi$.

We focus on the bosonic potential
\begin{equation}%
  \label{eq:potential}
  V_{\text{aFH}} \left(\Phi, B\right) = \tr \left[m^2 B \Phi + g {( B \Phi )}^2\right]
\end{equation}
with non-trivial local minima
\begin{subequations}%
  \label{eq:minima}
  \begin{align}
    B_0    & \equiv \mathrm{diag} \left(b_1, \ldots, b_N \right) \;,             \\
    \Phi_0 & \equiv \mathrm{diag} \left(\varphi_1, \ldots, \varphi_N \right) \;,
  \end{align}
\end{subequations}
where each pair of eigenvalues $(b_j, \varphi_j)$ must solve
\begin{subequations}%
  \label{eq:eigenvalues-conditions}
  \begin{align}
    m^2 b_j + 2g b_j^2 \varphi_j +\alpha        & = 0 \;,  \\
    m^2 \varphi_j + 2g b_j \varphi_j^2  + \beta & = 0 \;,  \\
    \textstyle\sum_{j} b_j                      & = 0 \;,  \\
    \textstyle \sum_j \varphi_j                 & = 0 \;,  \\
    \Delta_k                                    & > 0  \;.
  \end{align}
\end{subequations}
Einstein sum convention is not assumed in~\eqref{eq:eigenvalues-conditions}, and indexes run as $j \in \left\{ 1, \ldots, N \right\} $, $k \in \left\{ 5, \ldots, 2+2N \right\} $. The parameters $\alpha, \beta \in \mathbb{R}$ are Lagrange multipliers enforcing the traceless condition characteristic of $\mathfrak{su}(N)$ matrices, and $\Delta_k$ is the determinant of the leading $k \times k $ principal minor of the bordered Hessian of $V_{\text{aFH}}$. We stress these are sufficient, but not strictly necessary conditions. For instance, if $N=2$, then only one independent Casimir operator exists, implying $\tr [ {(B \Phi)}^2 ] \propto { [ \tr ( B \Phi ) ] }^2$. In such case, solving~\eqref{eq:eigenvalues-conditions} can be completely avoided by simply \textquote{completing the square} in~\eqref{eq:potential}.

The gauge symmetry breaking pattern we find is exactly the same as the adjoint Higgs from particle physics. First, $(B_0, \Phi_0)$ are always valued in the Cartan subalgebra $\bigoplus^{N-1} \mathfrak{u}(1)$ of $\mathfrak{su}(N)$, and have stabilizer $\mathfrak{h}_0 = \left\{ X \in \mathfrak{su}(N) \; ; \; \mathrm{ad}_{B_0} X = 0, \;\mathrm{ad}_{\Phi_0} X = 0\right\} $ in the minima moduli $\mathcal{V}_0$. In the trivial neighborhood, $\mathfrak{h}_0 \sim \mathfrak{su}(N)$. Away from it, gauge symmetry is spontaneously broken, $\mathfrak{h}_0 \subset \mathfrak{su}(N)$. Specifically, in a region where the multiplicity of all $(b_j, \varphi_j)$ is $n_j=1$ (non-degeneracy), then $\mathfrak{h}_0 \sim \bigoplus^{N-1} \mathfrak{u}(1)$, implying $N(N - 1)$ broken generators, and $N(N-1)$ massive gauge field components. In a region where $l \in \left\{1, \ldots, m\right\} $ labels distinct pairs of eigenvalues, and $1 < n_l \leq N \; ; \; \sum_l n_l = N$ (some degeneracy), then $\mathfrak{h}_0 \sim \bigoplus_l \mathfrak{su}(n_l) \oplus \bigoplus^{m-1} \mathfrak{u}(1)$, implying $N^2 - \sum_l n_l^2$ broken generators, thus $N^2 - \sum_l n_l^2$ massive gauge fields components. The region of complete degeneracy ($m=1$) is the trivial neighborhood, implying no broken generators, no massive gauge fields, and no symmetry breaking overall.

Our action functional~\eqref{eq:afh-action}, in the neighborhood of $(B_0, \Phi_0)$, assumes the effective form
\begin{subequations}%
  \label{eq:afh-eff-action}
  \begin{align}%
    \bar{S}_{\text{aFH}} \left[ \bar{\Xi} \right] & = \bar{s} \int \tr \left\{ D \xi \star D\varphi + \mathrm{ad}_{\Phi_0} D \xi \star A + \xi ( \Phi_0 + \varphi ) \left[ m^2 + g ( B_0 \Phi_0 \right. + \right.                                                      \nonumber   \\
                                                  & \left. + \left.  B_0 \varphi + b \Phi_0 + b \varphi - \xi \eta ) \vphantom{m^2} \right] \star \mathds{1} \right\}     \;,                \label{eq:afh-eff-action-exact}                                                       \\
                                                  & = \int \tr \left\{ \vphantom{m^2} \mathrm{ad}_{B_0} \mathrm{ad}_{\Phi_0} A \star A + \left( \mathrm{ad}_{B_0} \varphi + \mathrm{ad}_{\Phi_0} b \right) D \star A  + \mathrm{ad}_{\Phi_0} \xi D \star \psi \right. +  \nonumber \\
                                                  & - Db \star D \varphi - D \xi \star D \eta - \left( \mathrm{ad}_{B_0} \mathrm{ad}_{A} \varphi + \mathrm{ad}_{\Phi_0} \mathrm{ad}_{A} b \right) \star A + ( \mathrm{ad}_{\varphi} D \xi \; +  \nonumber                          \\
                                                  & - \mathrm{ad}_{\xi} D \varphi + 2 \mathrm{ad}_{\Phi_0}\mathrm{ad}_{A} \xi ) \star \psi + \left[ ( m^2 + 2g B_0 \Phi_0 ) ( b \varphi - \xi \eta ) + g \left( B_0 \varphi \right. + \right. \nonumber                            \\
                                                  & \left. + \left. \left.  b \Phi_0 + b \varphi - \xi \eta \right)^2 \right] \star \mathds{1} \right\} \;, \label{eq:afh-eff-action-full}                                                                                         % chktex 3
  \end{align}
\end{subequations}
where $\bar{S}_{\text{aFH}} \left[ \bar{\Xi} \right] \equiv S_{\text{aFH}} [A, \psi, \Phi_0 + \varphi, \eta, \xi, B_0 + b]$; $\bar{\Xi}$ is shorthand for $A, \psi, \varphi, \eta, \xi, b$, and; $\bar{s}$ is given by~\eqref{eq:afh-bar-brst}. In particular, its quadratic part
\begin{align}%
  \label{eq:afh-eff-quad-action}
  \bar{S}_{\text{aFH}}^{(2)} \left[ \bar{\Xi} \right] & = \int \tr \left\{ \vphantom{m^2} \mathrm{ad}_{B_0} \mathrm{ad}_{\Phi_0} A \star A + \left( \mathrm{ad}_{B_0} \varphi + \mathrm{ad}_{\Phi_0} b \right) d \star A + \mathrm{ad}_{\Phi_0} \xi d \star \psi \right. +  \nonumber \\
                                                      & - \left. db \star d\varphi - d\xi \star d\eta +\left[ ( m^2 + 2g B_0 \Phi_0 ) \left( b \varphi - \xi \eta \right) + g {\left( B_0 \varphi + b \Phi_0 \right)}^2 \right] \star \mathds{1} \right\}
\end{align}
makes very clear that some of the effective gauge fields have mass defined by the mass matrix $\mathrm{ad}_{B_0}\mathrm{ad}_{\Phi_0}$. This is, of course, a direct reflection of the spontaneous gauge symmetry breaking described above. The terms proportional to $d \star A$ and $d \star \psi$ are removable via gauge fixing. And, the massive bosonic pair $(b, \varphi)$ plays the role similar to the Higgs boson field, with $(\xi,\eta)$ being its massive fermionic partner (note that $(\xi,\eta)$ only kept its mass due to the factor $m^2+2gB_0\Phi_0$ not identically vanishing given the \textquote{adjointness} of our fields).

One can quickly check that no massless fermions are present in~\eqref{eq:afh-eff-action}, which indicates that the topological BRST symmetry is unbroken at tree-level. The effective BRST, given by
\begin{subequations}%
  \label{eq:afh-bar-brst}
  \begin{align}
    \bar{s} A    & =   \psi \;,   & \bar{s} \varphi & =   \eta \;,    & \hspace{200pt}                            \\
    \bar{s} \psi & =  - D\phi \;, & \bar{s} \eta    & = 0 \;,         & \hspace{200pt}                            \\
    \bar{s} \phi & = 0 \;,        & \bar{s} \xi     & =  B_0 + b  \;, & \hspace{200pt}\label{eq:afh-bar-brst-ssb} \\
    \bar{s} F    & = -D\psi       & \bar{s} b       & = 0      \;,    & \hspace{200pt}
  \end{align}
\end{subequations}
indeed remains nilpotent (up to $\delta_{\phi}$), and isomorphic to~\eqref{eq:cartan-model-afh-brst}. On one hand, the effective Higgs phase action~\eqref{eq:afh-eff-action-full} can indeed be written as the $\bar{s}$-boundary~\eqref{eq:afh-eff-action-exact}, which automatically makes it a $\bar{s}$-cycle\footnote{The use of the Cartan model is justified because the integrand in~\eqref{eq:afh-eff-action-exact} is basic. This is somewhat surprising given to the explicit presence of $A$ outside $D$. We attribute this to the tree-level stability of the BRST\@. As long as $s(B_0, \Phi_0) = (0,0)$, the basicness of $(B,\Phi)$ gets passed down to $(B_0, \Phi_0)$, and to $(b,\varphi)$, ultimately ensuring basicness of the integrand (and the nilpotency of $\bar{s}$ on it).}. On the other, if $ \langle 0 \rvert b \lvert 0 \rangle = 0$,~\eqref{eq:afh-bar-brst-ssb} implies that $\langle 0 \rvert \bar{s} \xi \lvert 0 \rangle = B_0 $. As long as $B_0 \neq 0$, this is a sufficient condition for spontaneous BRST symmetry breaking. In turn, Nambu-Goldstone fermionic states must necessarily be present~\cite{fujikawa1983a,witten1982a,witten1981a,salam1974a}. Since such degrees of freedom are not present at tree-level, they must be generated by loop corrections or non-perturbative effects. Saying it differently, the lack of a Nambu-Goldstone fermion in~\eqref{eq:afh-eff-action} implies that the Higgs phases of quantum aTYMH theories must necessarily be quantum mechanically non-trivial --- regardless of the choice of gauge fixing conditions. Simultaneously, the $\mathcal{G}$-equivariant cohomology complex of $\mathcal{A}$ no longer fully captures their complete set of physical states: these are not cohomological TQFTs.

It is worth analyzing the energy functional evaluated from~\eqref{eq:afh-action},
\begin{subequations}%
  \label{eq:afh-energy}
  \begin{align}
    \mathbb{E}_{\text{aFH}} \left[ \Xi ; r\right] & =  \int_{B^3_r} \tr \left\{ \mathcal{L}_{\lambda}c \star ( \mathrm{ad}_{\Phi} D \xi - \mathrm{ad}_{\xi} D \Phi) + s \left[ \vphantom{m^2} \mathcal{L}_{\lambda} \Phi \star D \xi + D \xi (\lambda\rfloor \star D \Phi) \right. + \right. \nonumber \\
                                                  & \left. + \left. \mathrm{ad}_\xi \star D \Phi (\lambda \rfloor A) + \xi \Phi [ m^{2} + g ( B \Phi - \xi \eta ) ] ( \lambda \rfloor \star \mathds{1} ) \right] \right\}                                                                              \\
                                                  & =  \int_{B^3_r}\tr \left[ \vphantom{m^2} \mathcal{L}_{\lambda} B \star D \Phi - \mathcal{L}_{\lambda} \Phi \star DB + \mathcal{L}_{\lambda} \eta \star D \xi - \mathcal{L}_{\lambda} \xi \star D \eta \right. + \nonumber                          \\
                                                  & + \left. ( \mathrm{ad}_{\Phi} \mathcal{L}_{\lambda} \xi + \mathrm{ad}_{\xi} \mathcal{L}_{\lambda} \Phi ) \star \psi - (\lambda \rfloor \mathfrak{L}_{\text{aFH}}) \vphantom{m^2} \right] \;,
  \end{align}
\end{subequations}
where $\mathcal{L}_{\lambda} \equiv d \lambda \rfloor + \lambda \rfloor d$ is the Lie derivative along $\lambda$, $\mathfrak{L}_{\text{aFH}}$ is the Lagrangian 4-form in~\eqref{eq:afh-action-full}. It is clearly bulk trivial --- the non-exact piece, containing two $c$-vertices, is removable via gauge fixing\footnote{The canonical energy-momentum 3-form of a non-Abelian gauge theory generally does not originate from a basic form. This makes the Kalkman model unavoidable, resulting in the resurrection of $c$. However, if temporal gauge conditions ($\lambda \rfloor A = \lambda \rfloor \psi =0$) are assumed, then $\mathcal{L}_{\lambda}c$ identically vanishes in the bulk.}. In this sense, the situation around the trivial vacuum is quite similar to pure TYM theories, as every classical field configuration in the bulk is part of the vacua moduli. Unsurprisingly, the same can be said about the effective energy functional $\bar{\mathbb{E}}_{\text{aFH}}$, evaluated from~\eqref{eq:afh-eff-action}. Nonetheless, due to the BRST instability at a quantum level, we fully expect $\langle 0 \rvert \bar{\mathbb{E}}_{\text{aFH}} \lvert 0 \rangle $ to be bulk non-trivial. From a perturbative QFT perspective, this scenario very closely resembles the Coleman-Weinberg mechanism~\cite{coleman1973a}. And, exemplifies the effective non-topological features of the Higgs phase of quantum aTYMH theories.

Finally, we are ready to make a few comments about the presence of solitonic degrees of freedom around $(B_0, \Phi_0)$. Recalling the adjoint Yang-Mills-Higgs model from particle physics, the inequivalent gauge vacua form the moduli of flat connections: homotopically equivalent to a point. Thus, the homotopy of the adjoint Higgs sector alone, here $\pi_2 (SU(N)/{U(1)}^{m-1}) \sim \mathbb{Z}^{m-1}$, is sufficient to determine that $m-1$ independent topological solitons of spatial co-dimension 3 stabilizes in $\mathbb{R}^4_{\infty}$. A Riemannian or Lorentzian structure then implies these are vertices or monopoles, respectively. However, in aTYMH theories the effective gauge vacua is the entire gauge moduli $\mathcal{A}_\mathcal{H}/\mathcal{H}$, most definitely not homotopic to a point. Contributions from all these gauge vacua can potentially destabilize Higgs sector solitons, and \textit{vice versa}. Thus, a careful analysis of the complete moduli $[\mathcal{A}_{\mathcal{H}} \times C^{\infty} (\mathrm{ad}P_{\mathcal{H}})] / \mathcal{H} $ seems unavoidable in the classical realm\footnote{This is a very hard problem, way beyond our scope.}. Notwithstanding, the quantum realm suffers from BRST instability, generically implying $\langle 0 \rvert \bar{\mathbb{E}}_{\text{aFH}} \lvert 0 \rangle \neq 0 $ in the bulk. This latter fact localizes the gauge vacua back onto the moduli of flat connections, ensuring that the Higgs sector vertices/monopoles are quantum mechanically stable.

\begin{table}[htpb]
  \caption{Grading of all TYM fields, and the representation-independent FH fields.}%
  \label{tab:gradings} \begin{tabular}{cccccccccccc} \toprule
    Field      & $A$ & $F$  & $c$ & $\psi$ & $\phi$ & $\Phi$ & $\eta$ & $\xi$ & $B$  \\
    \midrule
    Form rank  & 1   & 2    & 0   & 1      & 0      & 0      & 0      & 0     & 0    \\
    Ghost no.  & 0   & 0    & 1   & 1      & 2      & 0      & 1      & -1    & 0    \\
    Statistics & odd & even & odd & even   & even   & even   & odd    & odd   & even \\
    \bottomrule
  \end{tabular}
\end{table}

\end{document}
