\documentclass[../main.tex]{subfiles}

\begin{document}

\section{Fujikawa-Higgs with BRST instability}%
\label{sec:fund-higgs}

Now, let us consider the Fujikawa-Higgs doublet fields in  the fundamental representation of $SU(N)$. For that, we introduce the associated vector bundle $E \equiv P \times_{SU(N)} \mathbb{C}^N$, its endomorphism bundle $\mathrm{End}(E) \equiv P \times_{SU(N)} \mathrm{End}(\mathbb{C}^N)$, and automorphism bundle $\mathrm{Aut}(E) \equiv P \times_{SU(N)} \mathrm{Aut}(\mathbb{C}^N)$. Then $\mathcal{G}$ has faithful representation in $\mathrm{Aut}(E)$, and $\mathrm{Lie(\mathcal{G})}$ has it in $\mathrm{End}(E)$. The TYM fields $(A, c, \psi, \phi)$ are now elements in $C^{\infty}(\mathrm{End}(E) \otimes \bigwedge \mathbb{R}^4_{\infty} )$, while the Fujikawa-Higgs fields $(\Phi, \eta, \xi, B)$ are elements in $ C^{\infty}(E \otimes \bigwedge^0 \mathbb{R}^4_{\infty} )$.

The BRST algebra is given by
\begin{subequations}%
  \label{eq:ffh-fields}
  \begin{align}
    s \Phi & = - c \Phi + \eta \;, & s \Phi^{\dagger} & = \Phi^{\dagger} c + \eta^{\dagger} \;,            & \hspace{100pt} \\
    s \eta & = - c \eta  \;,       & s \eta^{\dagger} & =  - \eta^{\dagger} c                 \;,          & \hspace{100pt} \\
    s \xi  & = - c \xi + B \;,     & s \xi^{\dagger}  & = - \xi^{\dagger} c + B^{\dagger}              \;, & \hspace{100pt} \\
    s B    & = - c B  \;,          & s B^{\dagger}    & = B^{\dagger} c                  \;,               & \hspace{100pt}
  \end{align}
\end{subequations}
where ${}^{\dagger}$ is defined via the canonical Hermitian metric on $E$. As usual, on $\mathrm{End}(E)$-valued fields it means Hermitian adjoint\footnote{In our convention, all TYM fields are anti-Hermitian.}, on $E$-valued fields conjugate transpose, and on $\mathbb{C}$-valued fields just conjugate.

In full analogy with the adjoint model, we define the $SU(N)$ \textquote{Fujikawa-Higgs} (FH) model via the action
\begin{subequations}%
  \label{eq:ffh-action}
  \begin{align}%
    S_{\text{FH}} & \equiv s \int \mathfrak{Re}\left\{(D \xi)^{\dagger} \star D \Phi + {\xi}^{\dagger} \Phi \left[m^2 + g ( {B}^{\dagger} \Phi - {\xi}^{\dagger} \eta ) \right] \star \mathds{1} \right\} \;,           \\
                  & = \int \re \left\{ \vphantom{m^2} -{(DB)}^{\dagger} \star D \Phi - {(D \xi)}^{\dagger} \star D \eta + {(D \Phi)}^{\dagger} \star \psi \xi + {(D \xi)}^{\dagger} \star \psi \Phi \right. + \nonumber \\
                  & + \left. \left[ m^2 ( {B}^{\dagger} \Phi - {\xi}^{\dagger} \eta ) + g {( {B}^{\dagger} \Phi - {\xi}^{\dagger} \eta )}^{2} \right] \star \mathds{1}  \right\} \;.
  \end{align}
\end{subequations}
However, the resulting bosonic potential
\begin{equation}%
  \label{eq:ffh-potential}
  V_{\text{FH}} \left({B}^{\dagger}, \Phi\right) = \re \left[m^2 B^{\dagger} \Phi + g {( B^{\dagger} \Phi )}^2\right]
\end{equation}
is, unfortunately, not bounded from below (or above). Assuming $g \neq 0$, the only critical point,
\begin{subequations}%
  \label{eq:saddle-pts}
  \begin{align}
    \re ({B}^{\dagger}\Phi) & = -m^2/2g \;, \\
    \im ({B}^{\dagger}\Phi) & = 0 \;,
  \end{align}
\end{subequations}
is a saddle. There is no analog of~\eqref{eq:afh-eff-action}. In other words, with the Fujikawa-Higgs fields in the fundamental representation of $SU(N)$, and bosonic potential given by~\eqref{eq:ffh-potential}, TYM does not admit spontaneous symmetry breaking, and representation in terms of massive gauge fields.

\subsection{Realified model}%

One possibility to move forward is to consider the realification (decomplexification) of the fundamental representation of $SU(N)$. This is done by applying the realification functor $\mathcal{F}$ on $\mathbb{C}^N$, forgetting its $\mathbb{C}$-linear structure, and resulting in a real $2N$-dimensional vector space $ \mathbb{C}^N_{\mathbb{R}} \equiv \mathcal{F} (\mathbb{C}^N) $. The functor also realifies the morphisms, $ \mathrm{End} ( {\mathbb{C}}^N_{\mathbb{R}} ) \equiv \mathcal{F} [ \mathrm{End} ( {\mathbb{C}}^N ) ] $, $ \mathrm{Aut} ( {\mathbb{C}}^N_{\mathbb{R}} ) \equiv \mathcal{F} [ \mathrm{Aut}( {\mathbb{C}}^N ) ] $, and implies an embedding $SU(N) \hookrightarrow SO(2N)$~\cite{kostrikin1989a,budinich1988a}. In practice, if $z \in \mathbb{C}^N$, then $z_{\mathbb{R}} = [ \re (z), \im (z) ]^T \in \mathbb{C}^N_\mathbb{R}$, and if $Z \in \mathrm{End}(\mathbb{C}^N)$, then
\begin{equation}%
  \label{eq:realified-matrix}
  Z_{\mathbb{R}} = \begin{pmatrix} \re (Z) & -\im (Z) \\ \im (Z) & \re (Z) \end{pmatrix} \in \mathrm{End}(\mathbb{C}^N_\mathbb{R}) \;.
\end{equation}
From now one, we will consider the realified bundle $E_{\mathbb{R}} \equiv P \times_{SU(N)} \mathbb{C}^{N}_\mathbb{R}$. The TYM fields are realified, $(A_{\mathbb{R}}, c_{\mathbb{R}}, \psi_{\mathbb{R}}, \phi_{\mathbb{R}}) \in C^{\infty}(\mathrm{End}(E_{\mathbb{R}}) \otimes \bigwedge \mathbb{R}^4_{\infty} )$, and the Fujikawa-Higgs fields are also realified $(\Phi_{\mathbb{R}}, \eta_{\mathbb{R}}, \xi_{\mathbb{R}}, B_{\mathbb{R}}) \in C^{\infty}(E_{\mathbb{R}} \otimes \bigwedge^0 \mathbb{R}^4_{\infty} )$.

We stress that realified $SU(N)$ TYM (TYM$_{\mathbb{R}}$) theories are not the same as full $SO(2N)$ TYM theories. The latter has a much bigger moduli than the former, reflecting the more involved nature of $SO(2N)$ bundles over $\mathbb{R}^4_{\infty}$. In particular, TYM$_{\mathbb{R}}$ only accounts for $SO(2N)$ bundles with 1st Pontryjagin number given by $p_1 (E_{\mathbb{R}}) = -2 k$. Accordingly, the action functional is given by
\begin{equation}%
  \label{eq:real-tym-action}
  S_{\text{TYM}_{\mathbb{R}}} [A_{\mathbb{R}}] \equiv \int \tr_{\mathbb{R}} (F_{\mathbb{R}}^2) = - 8 \pi^2 p_1 \;,
\end{equation}
where $\tr_{\mathbb{R}} (\hphantom{F}) \equiv 2\re [ \tr (\hphantom{F}) ] $ is the realified trace. From now one, we will drop the realified notation $_{\mathbb{R}}$ on the fields to avoid cluttering.

The realified BRST algebra remains (mostly) the same,
\begin{subequations}%
  \label{eq:real-ffh-fields}
  \begin{align}
    s A    & = - Dc + \psi \;,         & s \Phi & = - c \Phi + \eta \;, & s \Phi^{T} & = \Phi^{T} c + \eta^{T} \;,            & \hspace{0pt} \\
    s c    & = - c^2 + \phi  \;,       & s \eta & = - c \eta  \;,       & s \eta^{T} & =  - \eta^{T} c                 \;,    & \hspace{0pt} \\
    s \psi & = -D \phi - [c, \phi] \;, & s \xi  & = - c \xi + B \;,     & s \xi^{T}  & = - \xi^{T} c + B^{T}              \;, & \hspace{0pt} \\
    s \phi & = - [c, \phi]  \;,        & s B    & = - c B  \;,          & s B^{T}    & = B^{T} c                  \;,         & \hspace{0pt}
  \end{align}
\end{subequations}
where ${}^{T}$ is defined via the Euclidean metric on $E_{\mathbb{R}}$. As usual, on $\mathrm{End}(E_{\mathbb{R}})$-valued fields it means adjoint\footnote{TYM$_{\mathbb{R}}$ fields are antisymmetric.}, on $E_{\mathbb{R}}$-valued fields transpose, and on $\mathbb{R}$-valued fields it means identity.

The realified Fujikawa-Higgs (FH$_{\mathbb{R}}$) model is given by the action
\begin{subequations}%
  \label{eq:real-ffh-action}
  \begin{align}%
    S_{\text{FH}_{\mathbb{R}}} & = s \int \left\{ {(D \xi)}^{T} \star D \Phi + {\xi}^{T} \Phi \left[m^2 + g ( {B}^{T} \Phi - {\xi}^{T} \eta ) \right] \star \mathds{1} \right\} \;,                      \\
                               & = \int \left\{ \vphantom{m^2} -{(DB)}^{T} \star D \Phi - {(D \xi)}^{T} \star D \eta + {(D \Phi)}^{T} \star \psi \xi + {(D \xi)}^{T} \star \psi \Phi \right. + \nonumber \\
                               & + \left. \left[ m^2 ( {B}^{T} \Phi - {\xi}^{T} \eta ) + g {( {B}^{T} \Phi - {\xi}^{T} \eta )}^{2} \right] \star \mathds{1}  \right\} \;,
  \end{align}
\end{subequations}
and has bosonic potential
\begin{equation}%
  \label{eq:ffh-potential}
  V_{\text{FH}_{\mathbb{R}}} \left({B}^{T}, \Phi\right) = m^2 B^{T} \Phi + g {( B^{T} \Phi )}^2\right] \;,
\end{equation}
bounded from below by the non-trivial minima
\begin{subequations}%
  \label{eq:real-ffh-minima}
  \begin{align}
    {B}_0    & = (b_0, 0, \ldots, 0)^{T}  \;,       \\
    {\Phi}_0 & = (\varphi_0, 0, \ldots, 0)^{T}  \;,
  \end{align}
\end{subequations}
where $b_0 \varphi_0 = - m^2/2g \; \forall \; g>0$.

The gauge symmetry breaking pattern is the not same as the usual Higgs model of particle physics due to realification. Consider the standard Higgs with non-trivial vacua $\Phi_0$, and stabilizer $\mathfrak{h}_0 = \left\{ X \in \mathfrak{su}(N) \; ; \; X \Phi_0 = 0\right\} $ in $ \mathcal{V}_0 $. The condition $X \Phi_0 = 0$ forces the 1st row and 1st column of $X$ to vanish, making it effectively of rank $N-1$. This is the usual symmetry breaking pattern: in the trivial vacuum neighborhood $\mathfrak{h}_0 \sim \mathfrak{su}(N)$, while around $\Phi_0$ we have $\mathfrak{h}_0 \sim \mathfrak{su}(N-1)$. Consequentially, $2N-1$ generators are broken, resulting in $2N-1$ massive gauge bosons. In the realified Higgs, however, $\Phi_0$ has stabilizer $\mathfrak{h}_0 = \left\{ X \in \mathfrak{su}(N) \hookrightarrow \mathfrak{so}(2N) \; ; \; X \Phi_0 = 0 \right\} $, where $X$ is of the form~\eqref{eq:realified-matrix}. If its 1st row and 1st column vanish, so does its $(N+1)$-th row and $(N+1)$-th column, making it effectively of rank $2N-2$. Therefore, realification changes the gauge symmetry breaking pattern from $\mathfrak{su}(N) \mapsto \mathfrak{su}(N-1)$ to $\mathfrak{so}(2N) \mapsto \mathfrak{so}(2N-2)$. Now, $4N-3$ generators are broken, equating to $4N-3$ massive gauge boson. Clearly, realification amplifies the gauge symmetry breaking, increasing by $2N-2$ the number of massive gauge bosons\footnote{Similarly to how the embedding $S^1 \hookrightarrow \mathbb{R}^3$ gives $S^1$ more room to knot itself, the realification embedding $SU(N) \hookrightarrow SO(2N)$ gives gauge theories more gauge directions to break.}.

Back to FH$_{\mathbb{R}}$, the non-trivial vacua $(B_0,\Phi_0)$ has stabilizer $\mathfrak{h}_0 = \left\{ X \in \mathfrak{su}(N) \hookrightarrow \mathfrak{so}(2N) \; ; \; X B_0 = 0 \;, X \Phi_0 = 0 \right\} $, where $X$ is of the form~\eqref{eq:realified-matrix}. As we just learned, this means a gauge symmetry breaking pattern $\mathfrak{so}(2N) \mapsto \mathfrak{so}(2N-2)$, resulting in $4N-3$ massive gauge boson. The effective action around this vacua is
\begin{align}%
  % \label{eq:eff-action}
  \bar{S}_{\text{FH}_\mathbb{R}} & = \int \tr \left\{ \vphantom{m^2} \mathrm{ad}_{B_0} \mathrm{ad}_{\Phi_0} A \star A + \left( \mathrm{ad}_{B_0} \phi + \mathrm{ad}_{\Phi_0} b \right) D \star A  \right. +  \nonumber           \\
                                 & - \left( \mathrm{ad}_{B_0} \mathrm{ad}_{A} \phi + \mathrm{ad}_{\Phi_0} \mathrm{ad}_{A} b \right) \star A + \mathrm{ad}_{\Phi_0} \xi D \star \psi + ( \mathrm{ad}_{\phi} D \xi \; +  \nonumber \\
                                 & - \mathrm{ad}_{\xi} D \phi + 2 \mathrm{ad}_{\Phi_0}\mathrm{ad}_{A} \xi ) \star \psi - Db \star D \phi - D \xi \star D \eta \; + \nonumber                                                     \\
                                 & + \left[ ( m^2 + 2g B_0 \Phi_0 ) \left( b \phi - \xi \eta \right) + g {\left( B_0 \phi + b \Phi_0 + b \phi \right)}^2 \right. +  \nonumber                                                    \\
                                 & \left. - \left. 2g \left(B_0 \phi + b \Phi_0 + b \phi \right) \xi \eta \vphantom{m^2} \right] \star \mathds{1} \right\}
\end{align}
and has quadratic part
\begin{align}%
  % \label{eq:eff-quad-action}
  \bar{S}^{(2)}_{\text{FH}_\mathbb{R}} & = \int \tr \left\{ \vphantom{m^2} \mathrm{ad}_{B_0} \mathrm{ad}_{\Phi_0} A \star A + \left( \mathrm{ad}_{B_0} \phi + \mathrm{ad}_{\Phi_0} b \right) d \star A + \mathrm{ad}_{\Phi_0} \xi d \star \psi \right. +  \nonumber \\
                                       & - \left. db \star d\phi - d\xi \star d\eta +\left[ ( m^2 + 2g B_0 \Phi_0 ) \left( b \phi - \xi \eta \right) + g {\left( B_0 \phi + b \Phi_0 \right)}^2 \right] \star \mathds{1} \right\} \;.
\end{align}
Clearly, $\mathrm{ad}_{B_0}\mathrm{ad}_{\Phi_0}$ is the mass matrix acquired by $A$ due to the spontaneous gauge symmetry breaking. The bosonic pair $(b, \phi)$ plays the role of the Higgs boson field and remains massive. The fermionic pair $(\eta,\xi)$ also remains massive and indicates the BRST is not spontaneously broken.

The lack of minim relevant moduli is $\mathcal{M} = [\mathcal{A} \times C^{\infty}(E)]/\mathcal{G}$

The second and third term can be eliminated via the (A)SDL gauge fixing. Mean
\begin{subequations}%
  % \label{eq:fh-energy}
  \begin{align}
    \mathbb{E}_{\text{FH}} \left[\ldots ; r\right] & = \int_{B^3_r} \tr \left\{ \vphantom{m^2} \mathcal{L}_{\lambda} c \star \left( \mathrm{ad}_\Phi  D \xi -  \mathrm{ad}_{\xi} D \Phi \right) + s \left[ \vphantom{m^2} \mathcal{L}_{\lambda} \Phi \star D \xi + D \xi \left(\lambda\rfloor \star D \Phi\right) + \right. \right.\nonumber \\
                                                   & \left. + \left. \left(\lambda \rfloor A\right) \mathrm{ad}_\xi \star D \Phi + ( \lambda \rfloor \star \mathds{1} ) \xi \Phi [ m^{2} + g \left( B \Phi - \xi \eta \right) ] \right] \right\}                                                                                             \\
                                                   & =  \int_{B^3_r}\left\{ \tr \left[ \vphantom{m^2} \mathcal{L}_{\lambda} \eta \star D \xi - \mathcal{L}_{\lambda} \Phi \star DB + D \xi \left( \lambda \rfloor \star D \eta \right) - DB \left( \lambda \rfloor \star D \Phi \right) \right. + \right.\nonumber                           \\
                                                   & + \left(\lambda \rfloor A \right)  \left[ \mathrm{ad}_\xi \star D \eta + \mathrm{ad}_B \star D \Phi + \left(\mathrm{ad}_\Phi \mathrm{ad}_\xi + \mathrm{ad}_\xi \mathrm{ad}_\Phi \right) \star \psi\right] + \nonumber                                                                   \\
                                                   & - \left. \left( \lambda \rfloor \star \psi\right) \left( \mathrm{ad}_{\Phi} D \xi - \mathrm{ad}_\xi D \Phi  \right) - ( \lambda \rfloor \star \mathds{1} ) (m^2 \xi \eta + 2g B \Phi \xi \eta ) \right] + \nonumber                                                                     \\
                                                   & + \left.  ( \lambda \rfloor \star \mathds{1} ) V_{\text{FH}} \vphantom{m^2} \right\}
  \end{align}
\end{subequations}
where $\mathcal{L}_{\lambda} = d \lambda \rfloor + \lambda \rfloor d$ is the Lie derivative along $\lambda$. In can be shown that $\mathcal{L}_{\lambda} c = 0$ if $\lambda \rfloor A = 0$ and $ \lambda \rfloor \psi = 0$ (temporal gauge).

\begin{equation}%
  % \label{eq:tymfh-action}
  S \left[A, \psi, \Phi, \xi, \eta, B \right] = S_{\text{TYM}} + S_{\text{FH}}
\end{equation}

The energy functional
\begin{equation}%
  % \label{eq:energy}
  \mathbb{E} = \mathbb{E}_{\text{TYM}} + \mathbb{E}_{\text{FH}}
\end{equation}

\end{document}
