\documentclass[main.tex]{subfiles}

\begin{document}
%\begin{titlepage}
\thispagestyle{empty}
%%%%%%%%%
% TITLE %
%%%%%%%%%
{
  \noindent
  \Large
  \bfseries
  Spontaneous gauge and BRST symmetry breaking in topological Yang-Mills theories
  \bigskip
  \bigskip
}

%%%%%%%%%%%
% AUTHORS %
%%%%%%%%%%%
%\begin{addmargin}[20mm]{0mm}
{
  \noindent
  \bfseries
  Guilherme~Sadovski
  \bigskip
}

%%%%%%%%%%%%%%%
% AFFILIATION %
%%%%%%%%%%%%%%%
{
  \noindent
  \footnotesize
  Affiliation~1.

  \bigskip
}

%%%%%%%%%%
% EMAILS %
%%%%%%%%%%
{
  \noindent
  \footnotesize
  \rmfamily
  e-mails: \href{mailto:gsadovski@proton.me}{gsadovski@proton.me}
  \bigskip
}

%%%%%%%%%%%%
% ABSTRACT %
%%%%%%%%%%%%
{
  \noindent
  \bfseries
  Abstract.\normalfont{} We show that topological Yang-Mills theories admit representations in terms of massive gauge fields, monopoles, and textures, in a non-trivial neighborhood of their moduli. Spontaneous BRST symmetry breaking is absent in the adjoint representation, but present in fundamental one. The latter models indicate that local degrees of freedom can appear in the physical spectrum of topological field theories. Additionally, they provide examples of BRST instability in non-Abelian gauge theories with trivial Gribov problem.

  % Abstract.\normalfont{} We propose a consistent coupling between topological Yang-Mills (TYM) theories and Higgs-like doublets. We show that TYM admit representations in terms of massive gauge fields, monopoles, textures, \textit{etc.}, due to spontaneous gauge symmetry breaking. Spontaneous BRST symmetry breaking is absent in the adjoint models, but present in fundamental ones. The latter hints local degrees of freedom are reintroduced in the physical spectrum of these theories, and might inspire approaches which try to connect topological quantum field theories to local physics.

  \bigskip
}
%\end{addmargin}

%%%%%%%
% TOC %
%%%%%%%
{
  \noindent
  \rule{\textwidth}{1pt}
  \tableofcontents
  \smallskip
  \noindent
  \rule{\linewidth}{1pt}
}
%\end{titlepage}
\end{document}
