\documentclass[../main.tex]{subfiles}

\begin{document}

\section{TYMH theories}%
\label{sec:fundamental}

We continue our investigation considering the Fujikawa-Higgs (FH) doublets in the fundamental representation of $SU(N)$. For that, we introduce the associated vector bundle $E \equiv P \times_{SU(N)} \mathbb{C}^N$, its endomorphism bundle $\mathrm{End}(E) \equiv P \times_{SU(N)} \mathrm{End}(\mathbb{C}^N)$, and automorphism bundle $\mathrm{Aut}(E) \equiv P \times_{SU(N)} \mathrm{Aut}(\mathbb{C}^N)$. The group of gauge transformations $\mathcal{G}$ has faithful representation in $\mathrm{Aut}(E)$, and its Lie algebra $\mathrm{Lie(\mathcal{G})}$ has it in $\mathrm{End}(E)$. The TYM fields $(A, c, \psi, \phi)$ are now elements in $C^{\infty}(\mathrm{End}(E) \otimes \bigwedge \mathbb{R}^4_{\infty} )$, while the FH doublets $(\Phi, \eta, \xi, B)$ are in $ C^{\infty}(E \otimes \bigwedge^0 \mathbb{R}^4_{\infty} )$. Again, the gradings can be found in Table~\ref{tab:gradings}.

The BRST algebra is given by
\begin{subequations}%
  \label{eq:ffh-fields}
  \begin{align}
    s \Phi & = - c \Phi + \eta \;, & s \Phi^{\dagger} & = \Phi^{\dagger} c + \eta^{\dagger} \;,            & \hspace{100pt} \\
    s \eta & = - c \eta  \;,       & s \eta^{\dagger} & =  - \eta^{\dagger} c                 \;,          & \hspace{100pt} \\
    s \xi  & = - c \xi + B \;,     & s \xi^{\dagger}  & = - \xi^{\dagger} c + B^{\dagger}              \;, & \hspace{100pt} \\
    s B    & = - c B  \;,          & s B^{\dagger}    & = B^{\dagger} c                  \;,               & \hspace{100pt}
  \end{align}
\end{subequations}
where ${}^{\dagger}$ is defined via the canonical Hermitian metric on $E$. As usual, ${}^{\dagger}$ on $\mathrm{End}(E)$-valued fields means Hermitian adjoint\footnote{In our convention, all TYM fields are anti-Hermitian.}, on $E$-valued fields it means conjugate transpose, and on $\mathbb{C}$-valued fields it means complex conjugate.

In full analogy with the adjoint formulation, we define the Fujikawa-Higgs sector of \textquote{topological Yang-Mills-Higgs} (TYMH) theories via the action functional
\begin{subequations}%
  \label{eq:ffh-action}
  \begin{align}%
    S_{\text{FH}} \left[ \Xi \right] & \equiv s \int \mathfrak{Re}\left\{{(D \xi)}^{\dagger} \star D \Phi + {\xi}^{\dagger} \Phi \left[m^2 + g ( {B}^{\dagger} \Phi - {\xi}^{\dagger} \eta ) \right] \star \mathds{1} \right\} \;,         \\
                                     & = \int \re \left\{ \vphantom{m^2} -{(DB)}^{\dagger} \star D \Phi - {(D \xi)}^{\dagger} \star D \eta + {(D \Phi)}^{\dagger} \star \psi \xi + {(D \xi)}^{\dagger} \star \psi \Phi \right. + \nonumber \\
                                     & + \left. \left[ m^2 ( {B}^{\dagger} \Phi - {\xi}^{\dagger} \eta ) + g {( {B}^{\dagger} \Phi - {\xi}^{\dagger} \eta )}^{2} \right] \star \mathds{1}  \right\} \;,
  \end{align}
\end{subequations}
where $\Xi$ is shorthand for $A, \psi, \Phi, \eta, \xi, B, {\Phi}^{\dagger}, {\eta}^{\dagger}, {\xi}^{\dagger}, {B}^{\dagger}$. However, our analysis is quite short-lived because the resulting bosonic potential,
\begin{equation}%
  \label{eq:ffh-potential}
  V_{\text{FH}} ( \Phi, B, {\Phi}^{\dagger}, {B}^{\dagger} ) = \re \left[m^2 B^{\dagger} \Phi + g {( B^{\dagger} \Phi )}^2\right] \;,
\end{equation}
is not bounded from below (or above, for that matter). Assuming $g \neq 0$, one can easily verify that the only critical point, namely,
\begin{subequations}%
  \label{eq:saddle-pts}
  \begin{align}
    \re ({B}^{\dagger}\Phi) & = -m^2/2g \;, \\
    \im ({B}^{\dagger}\Phi) & = 0 \;,
  \end{align}
\end{subequations}
is a saddle. Thus, there is no spontaneous BRST and/or gauge symmetry breaking since there is no analog of~\eqref{eq:afh-eff-action} and/or~\eqref{eq:afh-bar-brst}. Stated even more clearly, in the fundamental representation of $SU(N)$, and with bosonic potential given by~\eqref{eq:ffh-potential}, TYMH theories do not have effective symmetry broken phases.

\subsection{Realified TYMH theories}%

One possible way to move forward is to consider the realification (decomplexification) of the fundamental representation of $SU(N)$. This is done by applying the realification functor $\mathfrak{i}^*$ on $\mathbb{C}^N$, making it \textquote{forget} its $\mathbb{C}$-linear structure~\cite{kostrikin1989a,budinich1988a}. The result is a real $2N$-dimensional vector space $ \mathbb{C}^N_{\mathbb{R}} \equiv \mathfrak{i}^* (\mathbb{C}^N) $, with a linear complex structure $J$\footnote{In fact, the exists an isomorphism $f: (\mathbb{C}^{N}_{\mathbb{R}}, J) \rightarrow (\mathbb{R}^{2N},J')$ such that $J' \circ f = f \circ J$.}. The functor also realifies the morphisms, $ \mathrm{End} ( {\mathbb{C}}^N_{\mathbb{R}} ) \equiv \mathfrak{i}^* [ \mathrm{End} ( {\mathbb{C}}^N ) ] $, $ \mathrm{Aut} ( {\mathbb{C}}^N_{\mathbb{R}} ) \equiv \mathfrak{i}^* [ \mathrm{Aut}( {\mathbb{C}}^N ) ] $, and implies the canonical\footnote{Up to conjugacy in $SO(2N)$.} embedding $ \mathfrak{i}: SU(N) \hookrightarrow SO(2N)$. In practice, if $z \in \mathbb{C}^N$, then $z_{\mathbb{R}} = {[ \re (z), \im (z) ]}^T \in \mathbb{C}^N_\mathbb{R}$, and if $Z \in \mathrm{End}(\mathbb{C}^N)$, then
\begin{equation}%
  \label{eq:realified-matrix}
  Z_{\mathbb{R}} = \begin{pmatrix} \re (Z) & -\im (Z) \\ \im (Z) & \re (Z) \end{pmatrix} \in \mathrm{End}(\mathbb{C}^N_\mathbb{R}) \;.
\end{equation}
The particular form of~\eqref{eq:realified-matrix} ensures its commutativity with $J$ --- a sufficient condition for $Z_{\mathbb{R}}$ to indeed be the realification of $Z$.

From now one, we will consider the realified bundle $E_{\mathbb{R}} \equiv P \times_{SU(N)} \mathbb{C}^{N}_\mathbb{R}$. The TYM fields are realified, $(A_{\mathbb{R}}, c_{\mathbb{R}}, \psi_{\mathbb{R}}, \phi_{\mathbb{R}}) \in C^{\infty}(\mathrm{End}(E_{\mathbb{R}}) \otimes \bigwedge \mathbb{R}^4_{\infty} )$, and so are the FH fields $(\Phi_{\mathbb{R}}, \eta_{\mathbb{R}}, \xi_{\mathbb{R}}, B_{\mathbb{R}}) \in C^{\infty}(E_{\mathbb{R}} \otimes \bigwedge^0 \mathbb{R}^4_{\infty} )$. We stress that realified $SU(N) \hookrightarrow SO(2N)$ topological Yang-Mills (TYM$_{\mathbb{R}}$) theories are not the same as $SO(2N)$ TYM theories. The latter have much bigger moduli than the former, reflecting the more general nature of $SO(2N)$ bundles over $\mathbb{R}^4_{\infty}$. In fact, TYM$_{\mathbb{R}}$ theories only account for those bundles with 1st Pontryagin number $p_1 (E_{\mathbb{R}}) = -2 k$. Accordingly, their action functional is
\begin{equation}%
  \label{eq:real-tym-action}
  S_{\text{TYM}_{\mathbb{R}}} [A_{\mathbb{R}}] \equiv \int \tr_{\mathbb{R}} (F_{\mathbb{R}}^2) = - 8 \pi^2 p_1 \;,
\end{equation}
where $\tr_{\mathbb{R}} (\hphantom{F}) \equiv 2\re [ \tr (\hphantom{F}) ] $ is the realified trace.

The realified BRST algebra remains (mostly) the same\footnote{From now on, we omit the realified label \textquote{$_{\mathbb{R}}$} whenever there is a risk of cluttering equations.},
\begin{subequations}%
  \label{eq:real-brst}
  \begin{align}
    s A    & = - Dc + \psi \;,         & s \Phi & = - c \Phi + \eta \;, & s \Phi^{T} & = \Phi^{T} c + \eta^{T} \;,            & \hspace{0pt} \\
    s c    & = - c^2 + \phi  \;,       & s \eta & = - c \eta  \;,       & s \eta^{T} & =  - \eta^{T} c                 \;,    & \hspace{0pt} \\
    s \psi & = -D \phi - [c, \phi] \;, & s \xi  & = - c \xi + B \;,     & s \xi^{T}  & = - \xi^{T} c + B^{T}              \;, & \hspace{0pt} \\
    s \phi & = - [c, \phi]  \;,        & s B    & = - c B  \;,          & s B^{T}    & = B^{T} c                  \;,         & \hspace{0pt}
  \end{align}
\end{subequations}
where ${}^{T}$ is defined via the Euclidean metric on $E_{\mathbb{R}}$. As usual, ${}^{T}$ on $\mathrm{End}(E_{\mathbb{R}})$-valued fields means adjoint\footnote{In our conventions, all TYM$_{\mathbb{R}}$ fields are antisymmetric.}, on $E_{\mathbb{R}}$-valued fields it means transpose, and on $\mathbb{R}$-valued fields it means just the identity map.

The realified Fujikawa-Higgs (FH$_{\mathbb{R}}$) sector is given by the action
\begin{subequations}%
  \label{eq:real-ffh-action}
  \begin{align}%
    S_{\text{FH}_{\mathbb{R}}} \left[ \Xi_{\mathbb{R}} \right] & = s \int \left\{ {(D \xi)}^{T} \star D \Phi + {\xi}^{T} \Phi \left[m^2 + g ( {B}^{T} \Phi - {\xi}^{T} \eta ) \right] \star \mathds{1} \right\} \;,                      \\
                                                               & = \int \left\{ \vphantom{m^2} -{(DB)}^{T} \star D \Phi - {(D \xi)}^{T} \star D \eta + {(D \Phi)}^{T} \star \psi \xi + {(D \xi)}^{T} \star \psi \Phi \right. + \nonumber \\
                                                               & + \left. \left[ m^2 ( {B}^{T} \Phi - {\xi}^{T} \eta ) + g {( {B}^{T} \Phi - {\xi}^{T} \eta )}^{2} \right] \star \mathds{1}  \right\} \;,
  \end{align}
\end{subequations}
where $\Xi_{\mathbb{R}}$ is shorthand for $A, \psi, \Phi, \eta, \xi, \Phi^T, \eta^T, \xi^T, B^T $, and has bosonic potential
\begin{equation}%
  \label{eq:real-ffh-potential}
  V_{\text{FH}_{\mathbb{R}}} ( \Phi, B^T) = m^2 B^{T} \Phi + g {( B^{T} \Phi )}^2 \;,
\end{equation}
now, bounded from below by the non-trivial minima
\begin{subequations}%
  \label{eq:real-ffh-minima}
  \begin{align}
    {B}_0    & = {(b_0, 0, \ldots, 0)}^{T}  \;,       \\
    {\Phi}_0 & = {(\varphi_0, 0, \ldots, 0)}^{T}  \;,
  \end{align}
\end{subequations}
where $b_0 \varphi_0 = - m^2/2g \; \forall \; g>0$.

We find that the gauge symmetry breaking pattern differs from usual Higgs model from particle physics due to realification. Recalling the standard Higgs, with non-trivial vacua $\Phi_0$, and stabilizer $\mathfrak{h}_0 = \left\{ X \in \mathfrak{su}(N) \; ; \; X \Phi_0 = 0\right\} $ in $ \mathcal{V}_0 $; the condition $X \Phi_0 = 0$ forces the 1st row, and 1st column of $X$ to vanish, making it effectively of rank $N-1$. This is the usual symmetry breaking pattern: in the trivial vacuum neighborhood, $\mathfrak{h}_0 \sim \mathfrak{su}(N)$, while in the neighborhood of $\Phi_0$, we have $\mathfrak{h}_0 \sim \mathfrak{su}(N-1)$. Consequentially, $2N-1$ generators are broken, resulting in $2N-1$ massive gauge field components. Turning to realified Higgs, $\Phi_0$ has stabilizer $\mathfrak{h}_0 = \left\{ X \in \mathfrak{su}(N) \hookrightarrow \mathfrak{so}(2N) \; ; \; X \Phi_0 = 0 \right\}$, where $X$ must have the form~\eqref{eq:realified-matrix}. Notice that if its 1st row, and 1st column vanish, so does its $(N+1)$-th row, and $(N+1)$-th column, making it effectively of rank $2N-2$. Therefore, realification changes the gauge symmetry breaking pattern from $\mathfrak{su}(N) \mapsto \mathfrak{su}(N-1)$ to $\mathfrak{so}(2N) \mapsto \mathfrak{so}(2N-2)$. Since $4N-3$ generators are broken, $4N-3$ gauge field components are massive. Interestingly, realification amplifies\footnote{Roughly analogous to how an embedding $S^{1} \hookrightarrow \mathbb{R}^3$ gives the circle freedom to knot itself, a realification embedding $SU(N) \hookrightarrow SO(2N)$ gives Higgs theories more gauge directions to break.} the gauge symmetry breaking, increasing the number of massive gauge fields components by $2N-2$.

In TYMH$_{\mathbb{R}}$ theories, the neighborhood of the non-trivial vacuum $(B_0,\Phi_0)$ has stabilizer $\mathfrak{h}_0 = \left\{ X \in \mathfrak{su}(N) \hookrightarrow \mathfrak{so}(2N) \; ; \; X B_0 = 0, \; X \Phi_0 = 0 \right\} $, where $X$ has the form~\eqref{eq:realified-matrix}. We thus conclude their gauge symmetry breaking pattern is $\mathfrak{so}(2N) \mapsto \mathfrak{so}(2N-2)$. Consequentially, they admit massive representation around $(B_{0}, \Phi_{0})$ accounting for these $4N-3$ broken generators. Namely,
\begin{align}
  \label{eq:real-eff-action}
  \bar{S}_{\text{FH}_\mathbb{R}} \left[ {\bar{\Xi}}_{\mathbb{R}} \right] & = \int \left[ \vphantom{m^2} {B_0}^T A \star A \Phi_0 - {B_0}^T D \star A \varphi - {\Phi_0}^{T} D \star A b + {\Phi_0}^{T} D \star \psi \xi \right. +  \nonumber \\
                                                                         & - {(Db)}^{T} \star D \varphi - {(D \xi)}^{T} \star D \eta + {B_0}^{T} A \star A \varphi + {\Phi_0}^{T} A \star A b  \; +  \nonumber                               \\
                                                                         & + {(D \varphi)}^{T} \star \psi \xi + {(D \xi)}^{T} \star \psi \varphi - 2 {\Phi_0}^{T} A \star \psi \xi \; + \nonumber                                            \\
                                                                         & + \left. g {( {B_0}^{T} \varphi + b^T \Phi_0 + b^T \varphi - \xi^T \eta )}^2 \star \mathds{1} \right] \;,
\end{align}
with quadratic part given by
\begin{align}%
  \label{eq:real-eff-quad-action}
  \bar{S}^{(2)}_{\text{FH}_\mathbb{R}} \left[ {\bar{\Xi}}_{\mathbb{R}} \right] & = \int \left[ \vphantom{m^2} {B_0}^T A \star A \Phi_0 - {B_0}^T d \star A \varphi - {\Phi_0}^{T} d \star A b + {\Phi_0}^{T} d \star \psi \xi \right. +  \nonumber \\
                                                                               & - \left. d b^T \star d \varphi - d \xi^T \star d \eta +  g {( {B_0}^{T} \varphi + b^T \Phi_0 )}^2 \star \mathds{1} \right] \;,
\end{align}
where $\bar{S}_{\text{FH}_\mathbb{R}} [ {\bar{\Xi}}_{\mathbb{R}} ] \equiv S_{\text{FH}_\mathbb{R}} [ A, \psi, \Phi_0 + \varphi, \eta, \xi, b, \Phi^T_0 + \varphi^T, \eta^T, \xi^T, B^T_0 + b^T] $, and; ${\bar{\Xi}}_{\mathbb{R}}$ is shorthand for $A, \psi, \varphi, \eta, \xi, b, \varphi^T, \eta^T, \xi^T, b^T$.

Equations~\eqref{eq:real-eff-action} and~\eqref{eq:real-eff-quad-action} should immediately be contrasted with their analog~\eqref{eq:afh-eff-action} and~\eqref{eq:afh-eff-quad-action}, respectively, from the adjoint formulation. Among the similarities and differences, we notice the glaring absence in~\eqref{eq:real-eff-quad-action}, of the term proportional to $m^2 + 2g {B_0}^{T} \Phi_0$ --- which identically vanished given to the (realified) \textquote{fundamentalness} of our fields. As a result, $(\xi, \eta)$ is now massless while $(b,\phi)$ remains massive. This implies that their doublet structure is no longer realized in this region of the moduli. Dramatically, the BRST instability of TYMH$_{\mathbb{R}}$ theories is a tree-level phenomenon, and $(\xi, \eta)$ is nothing but the resulting tree-level Nambu-Goldstone fermion.

The effective BRST around $(B_0, \Phi_0)$, in the presence of tree-level instability, assumes the form
\begin{subequations}% 
  \label{eq:real-eff-brst}
  \begin{align}
    \bar{s} A    & = \psi \;,    & \bar{s} \varphi & = \eta - \bar{s} \Phi_0\;, & \hspace{200pt} \\
    \bar{s} c    & = \phi  \;,   & \bar{s} \eta    & = 0 \;,                    & \hspace{200pt} \\
    \bar{s} \psi & = -D \phi \;, & \bar{s} \xi     & = b + B_0 \;,              & \hspace{200pt} \\
    \bar{s} \phi & = 0  \;,      & \bar{s} b       & = - \bar{s} B_0  \;.       & \hspace{200pt}
  \end{align}
\end{subequations}
In comparison to its adjoint analog~\eqref{eq:afh-bar-brst}, it has acquired shifts which indeed spoil the doublet and basic nature of the (effective) FH fields. It remains nilpotent (up to $\delta_{\phi}$), but it is no longer isomorphic to~\eqref{eq:real-brst} in the subspace of basic forms. In other words,~\eqref{eq:real-eff-brst} does not represent the Cartan model of the $\mathcal{G}$-equivariant cohomology of $\mathcal{A}$. Even more dramatically,~\eqref{eq:real-eff-action} is not a $\bar{s}$-cycle, automatically implying it also not a $\bar{s}$-boundary.

The Higgs-phase effective theories defined by~\eqref{eq:real-eff-action} have lost every trace of topological BRST invariance --- the non-topological features hidden in the quantum regime of aTYMH theories, are here exposed at tree-level. In particular, the dodge of the doublet theorem by the FH fields, and the presence of $\mathrm{g}$ metric-contaminated terms in~\eqref{eq:real-eff-action}, together imply local (non-topological) degrees of freedom are now present and physically relevant in the bulk. In other words, in this region of the moduli, $\text{TYMH}_{\mathbb{R}}$ theories are clearly standard local field theories.

It is straightforward to verify that the energy functional $\mathbb{E}_{\text{TYMH}_{\mathbb{R}}}$, evaluated from~\eqref{eq:real-ffh-action} is bulk trivial, but $\bar{\mathbb{E}}_{\text{TYMH}_{\mathbb{R}}}$, evaluated from~\eqref{eq:real-eff-action}, is not. Again, due to the tree-level instability. Thus, the effective gauge vacua are already localized onto the moduli of flat connections. Despite this, solitonic degrees of freedom remain quite rare occurrences. The reason is that the quotients $SO(2N)/SO(2N-2)$ are diffeomorphic to Stiefel manifolds $V_2 (\mathbb{R}^{2N})$. Their non-trivial homotopy, $\pi_{2N-2} [ V_2 ( \mathbb{R}^{2N})] \sim \mathbb{Z} $, implies that only for $N=2$, we have a single topological soliton of spatial co-dimension 3 stabilizing in ${\mathbb{R}}_{\infty}^{4}$. A Riemannian or Lorentzian structure then manifests it as a vertex or monopole, respectively.

\end{document}
